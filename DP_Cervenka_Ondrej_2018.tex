% options:
% thesis=B bachelor's thesis
% thesis=M master's thesis
% czech thesis in Czech language
% slovak thesis in Slovak language
% english thesis in English language
% hidelinks remove colour boxes around hyperlinks

\documentclass[thesis=M,czech]{FITthesis}[2012/06/26]

\usepackage[utf8]{inputenc} % LaTeX source encoded as UTF-8

\usepackage{graphicx} %graphics files inclusion
% \usepackage{amsmath} %advanced maths
% \usepackage{amssymb} %additional math symbols

\usepackage{dirtree} %directory tree visualisation
\usepackage{minted}

% % list of acronyms
% \usepackage[acronym,nonumberlist,toc,numberedsection=autolabel]{glossaries}
% \iflanguage{czech}{\renewcommand*{\acronymname}{Seznam pou{\v z}it{\' y}ch zkratek}}{}
% \makeglossaries

\newcommand{\tg}{\mathop{\mathrm{tg}}} %cesky tangens
\newcommand{\cotg}{\mathop{\mathrm{cotg}}} %cesky cotangens

% % % % % % % % % % % % % % % % % % % % % % % % % % % % % % 
% ODTUD DAL VSE ZMENTE
% % % % % % % % % % % % % % % % % % % % % % % % % % % % % % 

\department{Katedra číslicového návrhu}
\title{Nadřazený systém pro správu garáže}
\authorGN{Ondřej} %(křestní) jméno (jména) autora
\authorFN{Červenka} %příjmení autora
\authorWithDegrees{Bc. Ondřej Červenka} %jméno autora včetně současných akademických titulů
\author{Ondřej Červenka} %jméno autora bez akademických titulů
\supervisor{Ing. Martin Daňhel}
\acknowledgements{Doplňte, máte-li komu a za co děkovat. V~opačném případě úplně odstraňte tento příkaz.}
\abstractCS{V~několika větách shrňte obsah a přínos této práce v~češtině. Po přečtení abstraktu by se čtenář měl mít čtenář dost informací pro rozhodnutí, zda chce Vaši práci číst.}
\abstractEN{Sem doplňte ekvivalent abstraktu Vaší práce v~angličtině.}
\placeForDeclarationOfAuthenticity{V~Praze}
\declarationOfAuthenticityOption{4} %volba Prohlášení (číslo 1-6)
\keywordsCS{Nahraďte seznamem klíčových slov v~češtině oddělených čárkou.}
\keywordsEN{Nahraďte seznamem klíčových slov v~angličtině oddělených čárkou.}
% \website{http://site.example/thesis} %volitelná URL práce, objeví se v tiráži - úplně odstraňte, nemáte-li URL práce

\begin{document}

\definecolor {codebg} {rgb} {0.92, 0.92, 0.92}

% \newacronym{CVUT}{{\v C}VUT}{{\v C}esk{\' e} vysok{\' e} u{\v c}en{\' i} technick{\' e} v Praze}
% \newacronym{FIT}{FIT}{Fakulta informa{\v c}n{\' i}ch technologi{\' i}}

\begin{introduction}
	%sem napište úvod Vaší práce
\end{introduction}

\chapter{Cíl práce}

\chapter{Analýza a návrh}

\section{Výběr komunikačního protokolu}

Nadřazený systém bude se svými klienty (monitorovací zařízení v~jednotlivých garážích) komunikovat přes WiFi nebo Ethernet. Základem komunikace bude TCP/IP protokol, je však potřeba zvolit vhodný protokol z~aplikační vrstvy OSI modelu, který na něm bude stavět.

\subsection{Vlastní protokol}

Jedna z~možností je implementovat vlastní protokol pomocí TCP/IP socketů. Toto řešení se mi však nezdá příliš vhodné, neboť nepřináší žádné významné výhody, naopak se s~ním pojí řada komplikací.

Pro vlastní protokol by bylo nutné vytvořit robustní server, který zvládá obsluhu více klientů najednou. Dále by vzhledem k~citlivosti přenášených dat bylo nutné implementovat nějakou formu šifrování. Tyto velmi obsáhlé problémy přitom řeší většina dnešních protokolů.

Další nevýhodou je nutnost implementace klientské části protokolu při vytváření nových zařízení spravovaných nadřazeným systémem. To do jisté míry omezuje jeho rozšiřitelnost.

\subsection{\texttt{HTTPS}}

Další možnost je využít ke komunikaci protokol HTTPS. V~tomto případě by klienti komunikovali se sytémem pomocí HTTP metod jako například \verb|get| nebo \verb|post|.

Jelikož součástí zadání je i webové uživatelské rozhraní, bude v~každém případě nutné použít webový server pro jeho provoz. Tento server by pak bylo možné využít i k~poskytnutí API pro komunikaci systému s~garážovými čidly.

HTTPS se také postará o~kryptografické zabezpeční přenášených dat, je však nutné získat certifikát k~ověření pravosti webového serveru. 

Zde lze použít například certifikáty nadace Let's Encrypt, které jsou zdarma. Kromě toho Let's Encrypt poskytuje také automatizačního klienta \textit{Certbot} \cite{certbot} pro snadné nasazení a aktualizaci jejich certifikátů.

Certifikát pro provoz HTTPS bude potřeba zajistit i v~případě, že komunikace s~klienty nebude postavena na tomto protokolu. Je totiž nutné také zabezpečit webové rozhraní, například kvůli ověření identity uživatele.

API realizované pomocí HTTPS je poměrně snadno rozšiřitelné. Pro nově implementovanou operaci stačí definovat URL a případně formát přenášených dat.

Další výhodou je snadná implementace na straně klienta, tedy garážového čidla. Knihovny umožňující vytvářet HTTP požadavky jsou dostupné většině populárních platforem jako například \textit{Arduino} \cite{ard_web} nebo \textit{ESP8266} \cite{esp_web}.

%\begin{listing}[htbp]
%\caption{\label{code:foo}Minted: Nyní ještě křupavější}
%\begin{minted}[bgcolor=codebg]{python}
%# ... code here ...
%\end{minted}
%\end{listing}

\subsection{Protokol \texttt{MQTT}}

\section{Výběr platformy}

Pro realizaci systému je nutné zvolit vhodnou platformu. Jelikož je cílem práce vytvořit fyzické zařízení, rozhodl jsem se jako základ použít některý z~jednodeskových počítačů, které jsou v~dnešní době na trhu. Tyto počítače bývají cenově velmi dostupné a zároveň poskytují dostatečný výkon a podporu pro provoz systému.

Při výběru počítače byla nejdůležitejším kritériem podpora softwaru potřebného k~implementaci monitorovacího systému. 

\chapter{Realizace}

\begin{conclusion}
	%sem napište závěr Vaší práce
\end{conclusion}

\bibliographystyle{csn690}
\bibliography{mybibliographyfile}

\appendix

\chapter{Seznam použitých zkratek}
% \printglossaries
\begin{description}
	\item[API]
	\item[HTTP] Graphical user interface
    \item[HTTPS] Graphical user interface
    \item[MQTT] Graphical user interface
    \item[OSI]
    \item[TCP/IP] Graphical user interface
    \item[URL]
\end{description}

\chapter{Obsah přiloženého CD}

%upravte podle skutecnosti

\begin{figure}
	\dirtree{%
		.1 readme.txt\DTcomment{stručný popis obsahu CD}.
		.1 exe\DTcomment{adresář se spustitelnou formou implementace}.
		.1 src.
		.2 impl\DTcomment{zdrojové kódy implementace}.
		.2 thesis\DTcomment{zdrojová forma práce ve formátu \LaTeX{}}.
		.1 text\DTcomment{text práce}.
		.2 thesis.pdf\DTcomment{text práce ve formátu PDF}.
		.2 thesis.ps\DTcomment{text práce ve formátu PS}.
	}
\end{figure}

\end{document}
