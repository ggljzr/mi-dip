\documentclass[thesis=M,czech]{templates/FITthesis}[2012/06/26]

\usepackage[utf8]{inputenc} % LaTeX source encoded as UTF-8

\usepackage{graphicx} %graphics files inclusion
% \usepackage{amsmath} %advanced maths
% \usepackage{amssymb} %additional math symbols
\usepackage{dirtree} %directory tree visualisation
\usepackage{minted}

\usemintedstyle{pastie}
\renewcommand\listingscaption{Ukázka}
\raggedbottom

% % list of acronyms
% \usepackage[acronym,nonumberlist,toc,numberedsection=autolabel]{glossaries}
% \iflanguage{czech}{\renewcommand*{\acronymname}{Seznam pou{\v z}it{\' y}ch zkratek}}{}
% \makeglossaries

\newcommand{\tg}{\mathop{\mathrm{tg}}} %cesky tangens
\newcommand{\cotg}{\mathop{\mathrm{cotg}}} %cesky cotangens

\department{Katedra číslicového návrhu}
\title{Nadřazený systém pro správu garáže}
\authorGN{Ondřej} %(křestní) jméno (jména) autora
\authorFN{Červenka} %příjmení autora
\authorWithDegrees{Bc. Ondřej Červenka} %jméno autora včetně současných akademických titulů
\author{Ondřej Červenka} %jméno autora bez akademických titulů
\supervisor{Ing. Martin Daňhel}
\acknowledgements{Děkuji panu  Ing. Martinu Daňhelovi za čas, který mi věnoval a zejména za cenné rady a odborné vedení mé diplomové práce.}
\abstractCS{
Tato diplomová práce se zabývá tvorbou zabezpečovacího systému pro správu garáží. Konkrétně jde o~návrh a implementaci nadřazeného systému, který sbírá a zpracovává data zaslaná podřízenými systémy v~jednotlivých garážích.

Výsledná aplikace komunikuje s~podřízenými systémy pomocí protokolu HTTPS a umožňuje správu garáží přes webové rozhraní. Celý nadřazený systém je možné provozovat na Raspberry Pi.
}
\abstractEN{This thesis deals with design and implementation of garage security and monitoring system, which collects data from subsystems placed in individual garages.

Result is an application that commmunicates with subsystems via HTTPS and allows garage monitoring and management trough web interface. Application can deployed on Raspberry Pi.
}
\placeForDeclarationOfAuthenticity{V~Praze}
\declarationOfAuthenticityOption{4} %volba Prohlášení (číslo 1-6)
\keywordsCS{bezpečnost, webová aplikace, Raspberry Pi}
\keywordsEN{security, web application, Raspberry Pi}
% \website{http://site.example/thesis} %volitelná URL práce, objeví se v tiráži - úplně odstraňte, nemáte-li URL práce

\begin{document}

\definecolor {codebg} {rgb} {0.92, 0.92, 0.92}
\definecolor {blue} {HTML} {0B6CBC}
\definecolor {blue2} {HTML} {6E6EDE}
\definecolor {red} {HTML} {E44F36}
\definecolor {green} {HTML} {068906}
\definecolor {cyan} {HTML} {068971}
\definecolor {magenta} {HTML} {BF1C75}

% \newacronym{CVUT}{{\v C}VUT}{{\v C}esk{\' e} vysok{\' e} u{\v c}en{\' i} technick{\' e} v Praze}
% \newacronym{FIT}{FIT}{Fakulta informa{\v c}n{\' i}ch technologi{\' i}}

\begin{introduction}
	Cílem této práce je vytvořit nadřazený systém pro monitorování garážového komplexu. Výsledná aplikace bude komunikovat pomocí WiFi či Ethernetu s~podřízenými systémy (zasílajícími údaje z~čidel v~garážích). Na základě získaných dat pak bude udržován stav jednotlivých garáží a vytvářena historii událostí.

Systém bude poskytovat webového rozhraní pro administraci. V~tom bude možné přidávat a odebírat podřízené systémy, zobrazovat jejich stav a zaznamenané události.

Vzhledem k~povaze zadání je nutné systém navrhnout s~ohledem na zabezpečení přenášených informací před odposloucháváním či manipulací. Též je nutné autorizovat uživatele přistupjící do webového rozhraní.

Dalším důležitým požadavkem je snadná rozšiřitelnost o~nové funkce. Systém by mělo být možné v~budoucnu doplnit o~možnost správy rozdílných podřízených systémů (například subsystémy pro sledování skladových zásob) či integraci s~mobilní aplikací. Bude tedy potřeba navrhnout vhodné API pro předávání informací mezi systémem a jeho klienty. 

V~této práci se chci zaměřit na tvorbu aplikace na jedné konkrétní hardwarové platformě, jako například Raspberry Pi. Aplikace spolu s~touto platformou by pak měla tvořit kompletní zařízení, které bude možné po základní konfiguraci (připojení do WiFi sítě, nastavení hesla) hned nasadit.

Výsledné řešení by však mělo být dostatečne nezávislé na zvolené platformě. Tudíž by neměl být problém spustit systém například na osobním počítači či virtuálním serveru.

V~analytické části (\ref{sec:an}) práce tedy přiblížím proces výběru vhodné platformy, komunikačního protokolu a dalších prvků systému. Také stručně popíšu podřízený systém (garážové čidlo), se kterým budu dále pracovat. Další části mapují návrh (\ref{sec:de}) zařízení na základě této analýzy, jeho implemetaci (\ref{sec:im}) a testování (\ref{sec:te}).
\end{introduction}

\chapter{Teoretický základ}
\label{sec:te}

\section{Základní pojmy}

\begin{itemize}
    \item uzivatel -- use case diagramy, nebo ty mozna az v navrhu
    \item klic
    \item api
    \item client/server
    \item tcp/ip
    \item osi
    \item http/https, pozadavek
    \item hash
    \item mqtt
    \item certifikát, k tomu veřejný a soukromý klíč
    \item vícevláknová obsluha
    \item apache
    \item nadrazeny system
    \item podrizeny system
    \item mitm attack
    \item python ??
    \item publisher/subscriber
    \item webové rozhraní
    \item unix time
\end{itemize}

\section{Bezpečnost}

\section{Spolehlivost}
\chapter{Analýza}
\label{sec:an}

\section{Struktura systému}
\label{sec:an_struct}

Struktura celého systému je naznačena na obrázku \ref{fig:basic_struct}. Podřízené systémy komunikují s~nadřazeným na základě událostí. Nadřazený systém tyto události zpracovává a upravuje podle nich stav garáží v~evidenci. 

Zaznamenané události jsou také uchovávány v~historii událostí, spolu s~dalšími metadaty jako čas přijetí nebo původce.

Komunikace mezi podřízeným a nadřazeným systémem je postavena na modelu \textit{client/server}. Nadřazený systém provozuje server zvoleného protokolu (viz sekce \ref{sec:an_protocol}), ke kterému se podřízené systémy připojují. Komunikaci tedy vždy iniciuje podřízený systém. S~možností zasílání nevyžádaných zpráv podřízeným systémům v~této práci nepočítám, mohl by to však být námět pro další rozšíření.

Další, kdo přistupuje do systému, je uživatel. Přes webové rozhraní může sledovat stav garáží a historii událostí. Také zde může spravovat klíče, které slouží pro přístup ke komunikačnímu API systému. Přístup do webového rozhraní je zabezpečen heslem.

\begin{figure}[h!]
    \centering
    \includegraphics[width=0.7\textwidth]{images/basic_struct.pdf}
    \caption{Základní struktura systému}
    \label{fig:basic_struct}
\end{figure}

\subsection{Podřízený systém}

Podřízený systém je zařízení umístěné v~každé garáži, které sleduje stav okolí pomocí těchto senzorových vstupů:

\begin{itemize}
    \item teplota,
    \item světelá intenzita (fotobuňka),
    \item detekce kouře,
    \item detekce pohybu,
    \item stav dveří.
\end{itemize}

V~případě překročení mezních hodnot se zařízení okamžitě hlásí nadřazenému systému. Kromě toho také v~pravidelných intervalech odesílá kontrolní hlášení. 

Vyhodnocení události je provedeno nadřazeným systémem. Podřízený systém tedy hlásí každou událost (například otevření dveří), aniž by nějak zkoumal její závažnost.

Základní požadavek na podřízený systém je schopnost komunikace přes Ethernet či WiFi pomocí protokolu zvoleného v~sekci \ref{sec:an_protocol}. Kromě toho může být hardware prakticky libovolný.

\section{Výběr komunikačního protokolu}
\label{sec:an_protocol}

Nejdřív je nutné určit způsob komunikace, který bude systém používat. Díky tomu se budu při vybíraní platformy moci ujistit, že jsou dostupné vhodné knihovny a další software. 

Nadřazený systém bude se svými klienty (monitorovací zařízení v~jednotlivých garážích) komunikovat přes WiFi nebo Ethernet. Základem komunikace bude TCP/IP protokol, je však potřeba zvolit vhodný protokol z~aplikační vrstvy OSI modelu, který na něm bude stavět.

Při výběru protokolu jsem vycházel z~předpokaldu, že systém bude provozován v~uzavřené síti a bez přístupu k~internetu. 

\subsection{Vlastní protokol}

Jedna z~možností je implementovat vlastní protokol pomocí TCP/IP socketů. Toto řešení se mi však nezdá příliš vhodné, neboť nepřináší žádné významné výhody, naopak se s~ním pojí řada komplikací.

Pro vlastní protokol by bylo nutné vytvořit robustní server, který zvládá obsluhu více klientů najednou. Dále by vzhledem k~citlivosti přenášených dat bylo nutné implementovat nějakou formu šifrování. Tyto velmi obsáhlé problémy přitom řeší většina dnešních protokolů.

Další nevýhodou je nutnost implementace klientské části protokolu při vytváření nových zařízení spravovaných nadřazeným systémem. To do jisté míry omezuje jeho rozšiřitelnost.

\subsection{HTTPS}
\label{sec:an_https}

Další možnost je využít ke komunikaci protokol HTTPS. V~tomto případě by klienti komunikovali se sytémem pomocí HTTP metod jako například \verb|get| nebo \verb|post|.

Jelikož součástí požadavků na systém je i webové uživatelské rozhraní, bude v~každém případě nutné použít webový server pro jeho provoz. Ten by pak bylo možné využít i k~poskytnutí API pro komunikaci systému s~garážovými čidly.

Vhodný webový server (jako například Apache) zajistí vícevláknovou obsluhu všech klientů. Protokol se také postará o~kryptografické zabezpeční přenášených dat, je však nutné získat certifikát k~ověření pravosti serveru (viz sekci \ref{sec:an_certs}).

Certifikát bude potřeba zajistit i v~případě, že komunikace s~klienty nebude postavena na tomto protokolu. Je totiž nutné také zabezpečit webové rozhraní, například kvůli ověření identity uživatele. Nutnost pořízení certifikátu tedy nepředstavuje nevýhodu oproti jiným protokolům. 

API realizované pomocí tohoto protokolu je poměrně snadno rozšiřitelné. Pro nově implementovanou operaci stačí definovat URL a případně formát přenášených dat.

Výhodou je také snadná implementace na straně klienta, tedy garážového čidla. Knihovny realizující klientskou část protokolu jsou dostupné na většině populárních platforem jako například Arduino (s~Ethernet \textit{shieldem}, oficiální knihovna EthernetClient \cite{ard_web}) nebo ESP8266 (knihovna esp8266wifi \cite{esp_web}).

\subsubsection{Certifikáty pro provoz HTTPS}
\label{sec:an_certs}

Pro provoz HTTPS serveru lze použít například certifikáty certifikační autority Let's Encrypt, které jsou poskytovány  zdarma. Kromě toho dodává Let's Encrypt také automatizačního klienta Certbot \cite{certbot} pro snadné nasazení a aktualizaci jejich certifikátů. Bohužel certifikáty jsou vydávány pouze na doménu \cite{lets_encrypt_faq}, což komplikuje použití v~místní síti.

Jiná možnost je použití \textit{self-signed} certifikátu. Tento certifikát není podepsaný žádnou certifikační autoritou, ale pouze vlastníkem certifikátu. Může tedy sloužit k~šifrování komunikace (poskytuje veřejný klíč), ale je zranitelný vůči \textit{man-in-the-middle} útoku \cite{cert_wallen}.

\textit{Self-signed} certifiát však lze použít k~šifrování komunikace na uzavřené lokální síti, za předpokladu, že je server s~certifikátem (přesněji s~jeho soukromým klíčem) dostatečně zabezpečen \cite{cert_wallen}. 

Nevýhodou tohoto řešení je nedůvěra webových klientů (certifikát není podepsán certifikační autoritou a nelze tedy ověřit jeho pravost), což by ovlivnilo přístup k~uživatelskému rozhraní a API systému. V~případě webového rozhraní by prohlížeč zobrazil varování o~neznámém certifikátu. To by však mohl uživatel ignorovat. 

Podřízené systémy by při zasílání požadavků museli přeskočit krok ověření totožnosti serveru. Jak toho dosáhnout například v~knihovně Requests pro Python je naznačeno v~ukázce \ref{lst:req_selfsigned}.

\begin{listing}[htbp]
\caption{\label{lst:req_selfsigned} Vytvoření HTTPS požadavku v~knihovně Requests, bez verifikace serveru}
\begin{minted}[bgcolor=codebg]{python}
>>> import requests
>>> r = requests.get('https://test.local/hello', verify=False)
>>> r.status_code
200
\end{minted}
\end{listing}

\subsubsection{Autentizace klientů na HTTPS}

Přístup k~API nadřazeného systému by měl být povolen pouze ověřeným klientům. Díky tomu bude možné zabránit například zasílání nepravdivých informací z~neznámých zdrojů.

Jednoduchou autentizaci přes HTTPS lze realizovat například pomocí generování API klíčů. Pro každý podřízený systém bude vygenerován klíč, kterým se při zasílání požadavku systém prokáže. Seznam platných klíčů by byl udržován v~databázi nadřazeného systému. Klíče by uživatel mohl přidávat nebo odebírat (například v~případě odcizení podřízeného systému) pomocí webového rozhraní.

Tyto klíče by také bylo nutné nahrát a uchovávat na podřízených systémech. Detaily tohoto procesu by záležely na platformě těchto systémů. Například u~Arduina by šlo klíč nahrát z~uživatelova počítače pomocí sériové linky (s~USB převodníkem) a udržovat ho v~EEPROM.

Také by bylo možné implementovat v~nadřazeném systému \uv{registrační mód}, který by bylo možné dočasně povolit ve webovém rozhraní. V~tomto módu by systém po přijetí speciálního API požadavku vygeneroval nový klíč. Ten by si uložil do své databáze platných klíčů, a také ho v~odpovědi zaslal žádajícímu zařízení. Pokud by mód povolen nebyl, odpovědel by systém chybovým kódem, například 403 -- \textit{Forbidden}. Zaslání požadavku z~podřízeného systému mohlo být provedeno stisknutím tlačítka.

Tento přístup by byl pravděpodobně uživatelsky příjemnější, přináší však potencionální bezpečnostní rizika. Například pokud by uživatel zapomněl mód vypnout, systém by byl otevřený k~registraci nežádoucích zařízení. Takový problém by se však dal řešit například automatickou deaktivací módu po uplynutí časového limitu.

Útočník snažící se získat klíč k~API by také mohl periodicky zkoušet registrační požadavek a čekat na aktivaci módu. Obrana proti tomuto útoku by byla složitější, šlo by například filtrovat IP adresy s~příliš častými požadavky.

Obecně vycházím z~toho, že i v~případě registrace nežádoucího zařízení nemůže toto zařízení krátkodobě způsobit výraznější škody -- do databáze nadřazeného systému může pouze zasílat nová data, která jsou navíc vázana k~jeho identitě (API klíči). Nemůže tedy získávat data od jiných podřízených systému či měnit jejich záznamy. Neautorizované zařízení se také objeví v~seznamu registrovaných API klíčů, kde může být snadno odhaleno. 

\subsection{MQTT}
\label{sec:an_mqtt}

MQTT je komunikační protokol založený na modelu \textit{publisher/subscriber}, určený pro použití v~prostředí s~omezenými zdroji (malý výkon procesoru, omezená paměť atd.) \cite{mqtt_valerie}.

Komunikace mezi jednotlivými klienty v~systému je zprostředkována pomocí centrály, nazývané \textit{broker}. Ta spravuje adresy -- \textit{topics} -- na kterých mohou klienti publikovat či odebírat zprávy.

\begin{figure}[h!]
    \centering
    \includegraphics[width=0.7\textwidth]{images/basic_mqtt.pdf}
    \caption[Příklad struktury protokolu MQTT]{Příklad struktury protokolu MQTT \cite{mqtt_eclipse}}
    \label{fig:basic_mqtt}
\end{figure}

Na obrázku \ref{fig:basic_mqtt} tedy klienti \textcolor{green}{A} a \textcolor{green}{B} začnou odebírat \textit{topic} \uv{Teplota}. Když pak klient \textcolor{magenta}{C} pak publikuje zprávu na tuto adresu, \textcolor{blue2}{\textit{broker}} se postará o~doručení všem odebírajícím klientům.

Adresy je možné hierarchicky strukturovat. Lze tedy tvořit skupiny, například \verb|/senzory/obyvak/teplota| nebo \verb|/senzory/kuchyne/vlhkost|. Zprávy je nutné publikovat na jednoznačnou adresu, při odebírání je však možné použít modifikátory \verb|+| a \verb|#| pro specifikování skupiny adres. Modifikátor \verb|+| odpovídá libovolnému jednomu stupni hierarchie, \verb|#| pak libovolnému počtu libovolných stupňů. Pro odebírání všech senzorů vlhkosti lze tedy použít adresu \verb|/senzory/+/vlhkost|. Všechna data by pak bylo možné odebírat na adrese \verb|/senzory/#|. \cite{mqtt_eclipse}

V~případě této práce by tedy jak nadřazený systém, tak podřízené systémy byly klienty \textit{brokeru}. Podřízené systémy by publikovaly naměřená data, která by nadřazený systém odebíral. Samotný \textit{broker} by pak mohl běžet souběžně s~nadřazeným systémem na zvolené platformě (například open-source \textit{broker} Mosquitto je dostupný na řadě platforem, včetně Raspberry Pi \cite{mqtt_mosquitto_wiki}).

Protokol podporuje tři možnosti QoS \cite{mqtt_valerie}:

\begin{itemize}
    \item Nejvýše jedno doručení -- tento mód pouze odešle zprávu, není zahrnut žádný opakovací mechanismus pro případ nedoručení.
    \item Alespoň jedno doručení -- v~tomto módu je zaručeno doručení zprávy, ta však může být doručena vícekrát.
    \item Přesně jedno doručení -- zde je ošetřeno i duplicitní doručování zpráv.
\end{itemize}

Použití sofistikovanějších metod doručení má vliv na výkon, a proto se v~některých případech vyplatí zvolit nižší úroveň QoS (například při posílání idempotentních zpráv). Pro tuto práci bych však pravděpodobně zvolil záruku přesně jednoho doručení.

\subsubsection{Šifrování a autentizace na MQTT}

V~této části se budu zabývat prostředky pro šifrování komunikace, které jsou dostupné v~\textit{brokeru} Mosquitto.

První možnost je pro zabezpečení komunikace využít certifikáty, podobně jako u~HTTPS. Zde by se pravděpodobně také využil \textit{self-signed} certifikát (blíže popsaný v~sekci \ref{sec:an_certs}). Mosquitto navíc také vyžaduje kořenový certifikát certifikační autority \cite{mqtt_mosquitto_tsl}. Při použití \textit{self-signed} ceritifikátů by bylo nutné tuto autoritu vytvořit a používané certifikáty u~ní podepsat (pro bližší informace viz \cite{ca_nguyen}). Kořenový certifikát by také bylo nutné distribuovat klientům.

Kromě certifikátů lze pro šifrování použít i PSK. V~tom případě \textit{broker} a jeho klienti používají pro zašifrování komunikace společný klíč (známý jak klientovi, tak \textit{brokeru}). Různí klienti přitom mohou mít různé klíče. \cite{mqtt_mosquitto_conf}

Bohužel podpora PSK v~MQTT klientech není příliš rozšířená. PSK je možné použít v~knihovně libmosquitto, určené pro C/C++ (s~vazbami pro Python). U~této knihovny se mi však podařilo najít pouze manuálovou stránku (viz \cite{libmosquitto_man}), bez informací o~jejím dalším vývoji či udržování. Modul poskytující vazby do Pythonu byl nicméně předán projektu Paho \cite{mosquitto_python}.

Paho poskytuje implementace MQTT klientů pro mnoho platforem (včetně například Arduina \cite{paho_embedded}). Dokumentace klientů pro C++ a Python však možnost šifrování pomocí PSK vůbec nezmiňuje \cite{paho_cpp_doc} \cite{paho_pyt_doc}.

Tyto možnosti lze použít i k~autentizaci klientů \textit{brokeru}. Při použití certifikátů lze v~konfiguračním souboru Mosquitta zvolit \verb|require_certificate| \cite{mqtt_mosquitto_conf}. Poté bude od klienta vyžadován certifikát prokazující jeho totožnost. Při použití PSK lze k~autentizaci využít sdílený klíč (\textit{broker} odmítne klienty s~neplatnými klíči) \cite{mqtt_mosquitto_conf}. Kromě toho je možno použít také autentizaci pomocí uživatelského jména a hesla, která je součástí MQTT protokolu, případně klienty neověřovat vůbec (a pouze šifrovat komunikaci) \cite{mqtt_mosquitto_conf}.

\subsection{Závěr výběru protokolu}

V~sekcích \ref{sec:an_https} a \ref{sec:an_mqtt} jsem se blíže podíval na dva poměrně rozšířené protokoly aplikační vrstvy, které by bylo možné použít pro tvorbu nadřazeného systému.

Pokud by mezi požadavky na systém bylo zahrnuto zasílání nevyžádaných zpráv podřízeným systému (jak je zmíněno v~sekci \ref{sec:an_struct}), zvolil bych pravděpodobně protokol MQTT. V~tom je tato funkcionalita velmi snadno implementovatelná -- stačí aby podřízené systémy odebíraly \textit{topic}, na kterém by nadřazený systém publikoval zprávy.

Jelikož se však v~této práci zabývám systémem, který zprávy pouze přijímá a zaznamenává, rozhodl jsem se pro HTTPS. Nasazení tohoto protokolu je o~něco snazší (není potřeba na zařízení instalovat \textit{broker} a zařizovat certifikační autoritu -- stačí \textit{self-signed} certifikát) a s~jeho použitím mám více zkušeností. Také se částečně uvolní požadavky na volbu platformy (webový server bude potřeba v~každém případě, při volbě HTTPS jako komunikačního protokolu mezi systémy tedy nebude nutný žádný další software). 

Každopádně bude mým cílem navrhnout výslednou aplikaci tak, aby rozhraní pro podřízené systémy realizované pomocí HTTPS bylo možné snadno nahradit MQTT rozhraním.

K~zabezpečení komunikace (včetně webového rozhraní) použiju \textit{self-signed} certifikát. Hlavní důvod je požadavek na použití v~místní síti, bez zaručeného přístupu k~internetu. Toto rozhodnutí nemá vliv na návrh a implementaci systému, pouze na jeho nasazení -- konkrétně konfiguraci webového serveru.

Pokud by provozovatel plánoval mít systém přístupný z~internetu (přes registrovanou doménu), musí v~každém případě k~zabezpečení použít certifikát podepsaný důvěryhodnou certifikační autoritou. Pak stačí pouze v~konfiguračním souboru webového serveru nahradit \textit{self-signed} certifikát podepsaným certifikátem. Není tedy nutné provádět změny v~kódu aplikace.

\section{Ukládání dat}

Zaznamenané události bude potřeba presistentně uchovávat. Zde by šly využít jednoduché textové logy, vhodnější však bude zvolit nějaký databázový systém -- například kvůli širším možnostem zpracování naměřených údajů.

Z~dostupných možností mě zaujal SQLite, což není klasický databázový stroj s~modelem \textit{client/server}, ale místo toho tvoří součást programu, který databázi používá. Přístup k~datům je realizován pomocí přímého čtení/zápisu do databázového souboru na disku.  Díky tomu má malé nároky na diskový prostor a operační paměť. \cite{sqlite_about}

Jelikož bude nadřazený systém pravděpodobně provozován na hardwaru s~omezenými zdroji, představují tyto vlastnosti nezanedbatelnou výhodu. Použití SQLite také zjednouší nasazení aplikace, neboť nebude nutné vytvářet a konfigurovat databázový server.

\section{Programovací jazyk pro tvorbu systému}

Pro tvorbu systému jsem se rozhodl použít programovací jazyk Python. S~tímto jazykem mám nejvíce zkušeností co se týče implementace webových aplikací. Je také dostatečně rozšířený, takže výsledný systém bude možné nasadit na poměrně širokém spektru platforem bez nutnosti složitějšího portování.

K~vytvoření webového rozhraní i API pro podřízené systémy jsem zvolil framework Flask. Hlavní důvod jsou opět předchozí zkušenosti s~tímto frameworkem. Flask také dává více volnosti při návrhu aplikace než například také velmi rozšířený framework Django.

\section{Výběr platformy}
\label{sec:an_plat}

Pro realizaci systému je nutné zvolit vhodnou platformu. Jelikož je cílem práce vytvořit fyzické zařízení, rozhodl jsem se jako základ použít některý z~jednodeskových počítačů, které jsou v~dnešní době na trhu. Tyto počítače bývají cenově velmi dostupné a zároveň poskytují dostatečný výkon a podporu pro provoz systému.

Při výběru počítače byla nejdůležitejším kritériem podpora softwaru potřebného k~implementaci monitorovacího systému. Na základě předchozí analýzy je tedy vyžadován následující software:

\begin{itemize}
    \item Webový server Apache2.
    \item Databázový systém SQLite.
    \item Programovací jazyk Python 3.
    \begin{itemize}
        \item Webový framework Flask.
    \end{itemize}
\end{itemize}

Pro provoz tohoto softwaru bude potřeba plnohodnotný operační systém, což vylučuje platformy využívající jednoduché mikrokontroléry, jako například Arduino Uno. Kromě toho je nutné připojení k~síti pomocí Ethernetu nebo WiFi. 

V~dalších sekcích jsem se blíže podíval na jednodeskové počítače Raspberry Pi (sekce \ref{sec:an_rpi}) a Zybo Zynq-7000 (sekce \ref{sec:an_zyb}) a zvážil jejich výhody a nevýhody pro implementaci systému.

\subsection{Raspberry Pi}
\label{sec:an_rpi}

\begin{figure}[h!]
    \centering
    \includegraphics[width=0.7\textwidth]{images/rpi.jpg}
    \caption[Raspberry Pi 3]{Raspberry Pi 3 (obrázek převzat z~\url{https://en.wikipedia.org/wiki/Raspberry_Pi})}
    \label{fig:rpi}
\end{figure}

Raspberry Pi je velmi rozšířený jednodeskový počítač. Jeho poslední verze, Raspberry Pi 3, je postavena na SoC Broadcom BCM2837 s~čtyřjádrovým procesorem ARM Cortex A53, který je až o~50 \% rychlejší než procesor předchozí verze \cite{rpi_benchoff}.

Dále nová verze přináší vlastní WiFi modul \cite{rpi_benchoff}, není tedy nutné se spoléhat na externí moduly. Kromě toho je možné počítač připojit k~síti pomocí Ethernetového portu. Ten je omezený na 100 Mb/s \cite{rpi_benchoff}, to by však vzhledem k~objemu dat přenášených mezi nadřazeným a podřízenými systémy nemělo představovat problém.

Deska také obsahuje čtyři USB a jeden HDMI port. Ty nejsou pro implementovaný systém zásadní, nicméně při počáteční konfiguraci zařízení (například nastavení WiFi hesla) může být připojení monitoru a klávesnice pro některé uživatele pohodlnější než použití SSH či sériové linky. Připojený monitor se také hodí při řešení problémů se startem operačního systému.

Raspberry Pi 3 bohužel nemá vlastní bateriově zálohovaný RTC obvod a k~udržování času využívá NTP \cite{rpi_rtc_ada}. K~tomu je však zapotřebí internetové připojení. Jelikož by zařízení mělo být možné používat i v~síti bez přístupu k~internetu, je nutné připojit externí RTC obvod, napřílad pomocí I2C sběrnice. Poté je možné systémové hodiny synchronizovat offline pomocí tohoto obvodu (místo NTP).

S~počítačem je možné použít množství operačních systémů, z~nichž nejrozšířenější je pravděpodobně Raspbian, linuxový systém postavený na Debianu. Pro ten jsou dostupné všchny potřebné softwarové balíčky popsané v~sekci \ref{sec:an_plat}. Jelikož počítač nemá žádné vlastní úložiště, je nutné operační systém provozovat na vložené SD kartě.

Jednou z~výhod tohoto počítače je obrovské množství podporovaných hardwarových periferií a knihoven pro ně. V~této práci pravděpodobně využiju pouze zmíněný RTC obvod, případné další rozšiřování systému (například o~vestavěný LCD displej) bude na Raspberry Pi pravděpodobně snadnější než na jiných platformách.

Kromě široké podpory je hlavní výhodou Raspberry Pi jeho cena. Poslední verze se pohybuje kolem 1200 Kč. K~celkovým nákladům na systém je ještě třeba připočítat cenu RTC obvodu a SD karty. Zde počítám s~použitím již připraveného modulu s~obvodem PCF8523. Ten vyjde asi na 200 Kč. Jako úložiště by měla plně dostačovat 16GB SD karta, která se dá pořídit za 200 Kč. Celkové náklady na hardware systému se tedy měly pohybovat kolem 1600 Kč.

\subsection{Zybo Zynq-7000}
\label{sec:an_zyb}

\begin{figure}[h!]
    \centering
    \includegraphics[width=0.7\textwidth]{images/zybo.jpg}
    \caption[Přípravek Zybo Zynq-7000]{Připravek Zybo Zynq-7000 (obrázek převzat z~\url{https://www.xilinx.com/products/boards-and-kits/1-4azfte.html})}
    \label{fig:zybo}
\end{figure}

Tento přípravek od společnosti Digilent je postavený na SoC Xilinx Zynq Z-7010. Hlavní předností tohoto čipu je kombinace dvoujádrového procesoru ARM Cortex A9 s~FPGA odpovídající sérii Artix-7 \cite{zybo_man}. 

Díky tomu je možná těsná integraci mezi aplikací běžící na procesoru a výkonnými, úzce specializovanými moduly, které jsou syntetizované na FPGA. Tento přístup, kdy je hardware a software systému vyvíjen souběžně se označuje jako \textit{hardware/software codesign}.

%dostupne operacni systemy -- petalinux, xilinux
Pro SoC ze série Zynq vznikla linuxová distribuce Xilinux, vycházející z~Ubuntu. Kromě plnohodnotného operačního systému (včetně například grafického rozhraní) poskytuje Xilinux také ovladače pro komunikaci s~FPGA pomocí AXI sběrnice \cite{xilibus}. K~tomu využívá IP jádro Xilibus, které funguje jako adaptér mezi procesorem a FPGA modulem (viz obrázek \ref{fig:xilibus}). Ten pak ke komunikaci může využívat standardni FIFO fronty a nemusí se zabývat AXI sběrnicí \cite{xilibus}.

\begin{figure}[h!]
    \centering
    \includegraphics[width=0.8\textwidth]{images/xilibus.pdf}
    \caption[Blokové schéma využití IP jádra Xilibus]{Blokové schéma využití IP jádra Xilibus \cite{xilibus}}
    \label{fig:xilibus}
\end{figure}

Jelikož Xilinux staví na Ubuntu (konkrétně na verzi 12.04 LTS \cite{xilibus}), neměl by být problém nainstalovat software potřebný pro provoz nadřazeného systému (viz sekce \ref{sec:an_plat}). 

Přípravek je možné připojiř k~síti pomocí Ethernetového portu, který podporuje rychlost až 1Gb/s \cite{zybo_man}. WiFi připojení by bylo možné realizovat pomocí USB modulu. Dále přípravek obsahuje HDMI a VGA port, audio konektory, čtyři tlačítka, čtyři přepínače a slot pro SD kartu \cite{zybo_man}.

Na přípravku je také k~dispozici 128 MB flash paměti \cite{zybo_man}, k~provozu tedy teoreticky není potřeba SD karta. Tato paměť by však pravděpodobně nestačila k~instalaci vhodného operačního systému a potřebného softwaru. I~zde by tedy bylo nutné použít SD kartu.

Stejně jako Raspberry Pi tato deska postrádá RTC obvod. Firma Digilent však dodává externí bateriově zálohované hodiny, které lze připojit pomocí Pmod rozhraní \cite{pmod_rtc_man}. 

Nevýhodu této desky (především v~porovnání Raspberry Pi) je její cena. Ta se pohybuje okolo 4000 Kč. K~tomu je nutné přičíst náklady na SD kartu a RTC obvod, případně i WiFi modul. Cena celého zařízení by se tedy pohybovala v~rozmezí 4500 až 5000 Kč.

%bacha tady este na "As of 9/21/2017- the Zybo will be replaced by the Zybo Z7-10. We have created this guide to help you migrate your designs to the Zybo Z7. " ale nevim jestli to ma cenu resit ty desky se lisej minimalne (SoC je stejnej)
%^^linx: https://store.digilentinc.com/zybo-z7-zynq-7000-arm-fpga-soc-development-board/

\subsection{Závěr výberu platformy}

\begin{table}[h!]
\centering
\begin{tabular}{|c | c | c |} 
 \hline
 & \textbf{Raspberry Pi 3} & \textbf{Zybo Zynq-7000} \\
 \hline 
 SoC & Broadcom BCM2837 & Xilinx Zynq Z-7010 \\ 
 Procesor & ARM Cortex A53 & ARM Cortex A9 \\
 & 4 jádra, 1,2 GHz & 2 jádra, 650 MHz \\
 RAM & 1024 MB LPDDR2 & 512 MB DDR3 \\
 FPGA & -- & Ekvivalent řady Artix-7 \\
 Úložiště & SD karta & 128 MB Flash, SD karta \\
 Síť & 100 Mb/s Ethernet, WiFi & až 1 Gb/s Ethernet \\
 Cena & 1200 Kč & 4000 Kč \\
 \hline
\end{tabular}
\caption[Srovnání platforem Raspberry Pi 3 a Zybo Zynq-7000]{Srovnání platforem Raspberry Pi 3 a Zybo Zynq-7000} \cite{rpi_benchoff} \cite{zybo_man}
\label{tab:plat_compare}
\end{table}

Jak vyplívá z~tabulky \ref{tab:plat_compare}, Raspberry Pi 3 poskytuje znatelně výkonnější procesor a více paměti za méně než třetinu ceny desky Zybo. Hlavní přidaná hodnota této platformy tedy spočívá v~integraci s~FPGA, ta má však v~případě této práce pouze velmi omezené využití. 

FPGA by bylo možné využít například k~šifrování úložiště, vzhledem k~předpokládáným objemům dat by však zrychlení oproti softwarovému šifrování neospravedlnilo vysokou cenu přípravku. Na druhou stranu vyšší výkon procesoru u~Raspberry Pi může mít pro nadřazený systém význam, například z~hlediska odezvy uživatelského rozhraní.

Jako platformu pro implementaci nadřazeného systému jsem tedy zvolil Raspberry Pi 3, především kvůli příznivé ceně, výrazně lepšímu poměru cena/výkon (pro tuto práci), a také kvůli podpoře a rozšiřitelnosti.

\subsection{Provoz aplikace na cloudové platformě}
\label{sec:an_cloud}

\begin{figure}[h!]
    \centering
    \includegraphics[width=0.8\textwidth]{images/aws.pdf}
    \label{fig:aws}
    \caption{Struktura systému při použití cloudu Amazon EC2}
\end{figure}

navrh k~servirovani
\chapter{Návrh}
\label{sec:de}

\section{Architektura MVC}

???

controller -- api

view -- webova stranka

model -- vyhodnocovani a logovani udalost, stavu garazi

???

viz \url{https://www.digitalocean.com/community/tutorials/how-to-structure-large-flask-applications#working-with-modules-and-blueprints-(components)}

\section{Model}

pouziti sqlalchemy

\subsection{Garáž}

\subsubsection{Stav garáže}

\subsection{Událost}

\subsubsection{Vyhodnocení události}

\section{Controller}

Flask API

\section{View}

\section{Autentizace}

\subsection{Autentizace uživatele}

\subsection{Autentizace podřízeného systému}

\subsubsection{Registrační mód}


\chapter{Implementace}
\label{sec:im}

V~této kapitole je popsána implementace nadřazeného systému podle návrhu z~kapitoly \ref{sec:de}. Jednotlivé sekce tedy popisují implementační detaily dílčích částí návrhového vzoru MVC. Poslední sekce \ref{sec:im_auth} se zabývá implementací autentizace uživatele.

\section{Struktura aplikace}

Aplikace nadřazeného systému je rozdělena do tří modulů:

\begin{itemize}
    \item \texttt{mod\_main} -- hlavní modul aplikace. V~tomto modulu je implementován \textit{model} systému popsaný v~sekci \ref{sec:de_model}. \textit{Model} je využíván i ostatními moduly (konkrétně modulem \texttt{mod\_api}). Dále tento modul implementuje webové rozhraní správy nadřazeného systému.
    \item \texttt{mod\_api} -- modul implementující API systému. Zde je implementován \textit{controller} ze sekce \ref{sec:de_api}, definující URL, pomocí kterých mohou podřízené systémy komunikovat s~nadřazeným systémem.
    \item \texttt{mod\_auth} -- modul implementující autentizaci uživatele při přihlašování do webového rozhraní.
\end{itemize}

K~rozdělení aplikace na jednotlivé moduly je použit nástroj frameworku Flask nazvaný \textit{blueprints} \cite{flask_blueprints}.  Hlavní důvod k~využití modulů je oddělení částí aplikace podle jejich funkce.

Při strukturování aplikace jsem vycházel z~článku \textit{How To Structure Large Flask Applications} \cite{flask_large}.

\section{Implementace \textit{modelu}}

Jak je zmíněno v~sekci \ref{sec:de_model}, \textit{model} aplikace je tvořen třídami \texttt{Garage} a \texttt{Event}. Tyto třídy přímo využívají databázi nadřazeného systému, k~jejich implementaci je proto použit framework SQLAlchemy \cite{sqlalchemy}, který výrazně usnadní práci s~databází.

Tento framework poskytuje přístup k~SQL databázím přímo z~jazyka Python, takže není nutné psát prakticky žádný SQL kód. Tabulku v~databázi je možné definovat jako Python třídu s~přísušnými atributy a SQLAlchemy vytvoří odpovídající databázové schéma. 

SQLAlchemy také umožňuje snadno definovat databázové vztahy. V~\textit{modelu} nadřazeného systému se vyskytuje pouze vztah $1:N$ mezi třídami \texttt{Garage} a \texttt{Event} (jedna garáž má mnoho událostí, každá událost má právě jednu garáž). Tento vztah je možno pomocí SQLAlchemy definovat následujícím způsobem:

\begin{listing}[htbp]
\caption{\label{lst:db_relationship} Vytvoření vztahu $1:N$ mezi třídami \texttt{Garage} a \texttt{Event}}
\begin{minted}[bgcolor=codebg]{python}
from sqlalchemy import Column, ForeignKey, Integer
from sqlalchemy.ext.declarative import declarative_base
from sqlalchemy.orm import relationship

Base = declarative_base()

# třídy Event a Garage
# dedi SQLAlchemy metody
# třídy Base
class Event(Base):
    ...
    # odkaz na příslušnou garáž
    garage_id = Column(Integer, ForeignKey(
        'Garage.id'), nullable=False)

class Garage(Base): 
    ...
    # definice 1:N vztahu mezi garáží a událostí
    events = relationship('Event', backref='Garage')
\end{minted}
\end{listing}

\subsection{Kontrola zmeškaných hlášení}

V~databázi nadřazeného systému je potřeba pravidelně provádět kontrolu, zda podřízené systémy zaslaly v~očekávaný čas kontrolní hlášení. Pokud bylo plánované hlášení promeškáno, je nutné změnit stav příslušné garáže na \uv{Nehlásí se}.

Aplikace nadřazeného systému nemá v~zásadě možnost, jak tuto kontrolu sama iniciovat, neboť pouze reaguje na příchozí požadavky (od uživatele či podřízeného systému). Provedení kontroly může být důsledkem takového požadavku, například pokud uživatel otevře hlavní stránku webového rozhraní. 

Provádět kontrolu hlášení pouze v~reakci na vnější podnět však není dostatečné. Pokud by nadřazený systém musel pro provedení kontroly hlášení čekat interakci s~webovým rozhraním (nebo například s~podřízeným systémem), nemuselo by vůbec dojít ke změně stavu garáže a tedy ani k~odeslání notifikačního e-mailu. Je tedy nutné zajistit pravidelné provádění kontrol na základě vnitřního podnětu.

Tento problém jsem vyřešil použítím plánovače APScheduler \cite{apscheduler}. APScheduler funguje jako knihovna do Pythonu, a umožňuje plánovat provádění zvolených funkcí. Nejde tedy o~externí program, plánovač je přímo součástí kódu nadřazeného systému \cite{apscheduler}. Vytvoření pravidelné kontroly hlášení podřízených systémů lze implementovat tímto způsobem:

\begin{listing}[htbp]
\caption{\label{lst:scheduler_check} Pravidelná kontrola hlášení podřízených systémů každých 5 minut pomocí knihovny APScheduler}
\begin{minted}[bgcolor=codebg]{python}

from apscheduler.schedulers.background import \
    BackgroundScheduler

# BackgroundScheduler běží v~samostatném vlákně,
# neblokuje tedy webovou aplikaci
scheduler = BackgroundScheduler()

# přidání pravidelného úkolu do plánovače
# Garage.check_reports je statická metoda
# třídy Garage, která provede kontrolu hlášení
# u~všech garáží v~databázi (a případně upraví jejich stav)
scheduler.add_job(Garage.check_reports, 'interval', minutes=5)
scheduler.start()
\end{minted}
\end{listing}

Plánovač se kromě kontroly hlášení hodí také při vypínání registračního módu. Ten z~bezpečnostních důvodů po aktivaci běží po dobu tří minut. Jeho vypnutí je naplánováno obdobně jako kontrola hlášení, jediný rozdíl je, že úkol není spouštěn v~pravidelném intervalu, ale pouze jednou.

\subsection{Zasíláním upozornění}

%posilani mailu pomoci toho eventu zmeny v sqlalchemy

%posilani mailu obecne jak je implementovany

\section{Implementace \textit{view}}

%jinja, makra

%wtforms

%jquery na filtry

\section{Implementace \textit{controlleru}}

%csrf, viz  napsat ze pro add garage a revoke key misto normalni routy, getu a linku pouzivame formulare a post kvuli vochrane pred csrf, viz \url{https://stackoverflow.com/questions/6812765/how-to-demonstrate-a-csrf-attack}. Timhle utokem by nekdo moh vytvaret garaze a rusit api klice, coz neni naka velka skoda ale spis na votravovani no (u~vostatnich veci (tj hlavne change password) to bylo uz driv v~pohode protoze byly pouzity ty flaskforms)

\subsection{API \textit{controller}}

%vyjimka z csrf (neni potreba login, prokazuje se klicem)

\section{Autentizace uživatele}
\label{sec:im_auth}

%asi hlavne bcrypt popsat

%login_required decorator \url{http://flask.pocoo.org/docs/0.12/patterns/viewdecorators/}

%session

%mozna csrf tady, napsat jak je tam zapnuty, ze default je soucasti toho flask_formu ale my ho chceme pro vsechny ty post routy a pak vyjimku pro api modul -- tady napsat jak se nam siknou ty blueprinty

%v~testovani je mozny tenhle utok demonstrovat a ukazat jak pekne nam to funguje (voproti ty prvni verzi kde byl na add garage normalne get). Ted se pri tim pokusu vo utok normalne zvobrazi invalid csrf token nebo tak neco. Pro porovnani zmen puvodni a vopraveny verze viz commit e52a54b2caa8eead85e8df28c738356a7541a1c4 (password redirect, to je posledni verze s~timhle bugem)

\begin{listing}[htbp]
\caption{\label{code:foo} Testovací listing}
\begin{minted}[bgcolor=codebg]{python}
# ... code here ...

import numpy as np

def foo(a):
    print(a)

class FooBar:
    def __init__(self):
        self.b = 10

a = [1, 2, 3, 4]
for i in a:
    if i == 2:
        print("hello world")
\end{minted}
\end{listing}
\chapter{Testování}
\label{sec:te}

\section{\textit{Unit} testy}

%tady jen strucne popsat co to je unit test at to nemusime cpat do teorie

\subsection{Testování \textit{modelu}}

\subsection{Testování API}

\subsection{Testování autentizace uživatele}

\begin{conclusion}
	
V~rámci práce byl vytvořen nadřazený systém pro zabezpečení garáží. Systém má za úkol sbírat zaznamenaná data od podřízených systémů monitorujících jednotlivé garáže a v~případě nebezpečí odeslat zprávu s~upozorněním. Uživatel (majitel či správce garážového komplexu) může systém spravovat pomocí webového rozhraní.

Aplikace nadřazeného systému byla vytvořena v~jazyce Python a nasazena na jednodeskovém počítači Raspberry Pi 3. Výsledkem práce je tedy jednoúčelové zařízení, které je po krátké konfiguraci možné hned použít.

Text práce lze rozdělit do pěti kapitol: teoretický úvod, analýza, návrh, implementace a testování. Teoretický úvod slouží k~definici použitého názvosloví. Kromě toho obsahuje stručný popis některých pojmů a konceptů, které se v~práci vyskytují.

V~analýze byla na základě zadání práce vytvořena specifikace nadřazeného systému, popisující jeho základní chování a požadavky, které má splňovat. Podle specifikace byl pak zvolen software a hardware vhodný k~implementaci.

Významný prostor byl věnován volbě protokolu, na kterém bude postavena komunikace mezi nadřazeným a podřízenými systémy. Analýza se blíže zaměřila na protokoly HTTPS a MQTT. Kromě základních funkcí obou protokolů byly zkoumány také možnosti zabezpečení a autentizace účastníků komunikace. Pro komunikaci s~podřízenými systémy byl nakonec zvolen protokol HTTPS.

Kromě protokolu byl v~analýze také zvolen způsob ukládání dat (databázový systém SQLite3) a odesílání upozornění. Právě zasílání upozornění se ukázalo jako poměrně zajímavý problém s~několika možnými řešeními. V~rámci analýzy byly otestovány internetové služby pro zasílání e-mailů (Gmail) a SMS (Twilio). Kromě toho bylo také otestováno zasílání SMS pomocí GSM modulu připojeného přes USB port. Tento způsob byl nakonec použit ve výsledném zařízení.

Pro implementaci aplikace nadřazeného systému byl vybrán jazyk Python a webový framework Flask. Hlavní důvod volby byla moje předchozí zkušenost s~těmito nástroji.

V~závěru analytické části byla zvolena vhodná hardwarová platforma pro tvorbu výsledného zařízení. Zkoumány byly jednodeskové počítače Raspberry Pi 3 a Zybo Zynq-7000. Deska Zynq-7000 byla zajímavá především díky integraci FPGA obvodu, ukázalo se však, že ten by měl v~práci pouze velmi omezené využití. Jako platforma pro tvorbu zařízení bylo tedy zvoleno Raspberry Pi 3, které poskytuje vyšší procesorový výkon za výrazně nižší cenu.

Na základě analýzy byl vytvořen návrh aplikace nadřazeného systému, kde byl jako základ použit vzor \textit{model-view-controller}. Díky tomuto vzoru by měl být systém poměrně snadno rozšiřitelný. Například přidání možnosti komunikovat s~podřízenými systémy pomocí protokolu MQTT lze realizovat pouze doplněním vhodného \textit{controlleru}.

Dále byl v~kapitole návrh popsán způsob vyhodnocování stavu monitorovaných garáží a odesílání upozornění.

Pro komunikaci s~podřízenými sytémy bylo navrženo API, pomocí kterého mohou nadřazenému systému zasílat události popisující stav garáže (například při detekci kouře). Podřízené systémy se při přístupu k~API prokazují pomocí náhodně generovaných klíčů.

Zde bylo potřeba vyřešit problém distribuce klíčů podřízeným systémům. K~tomu byl implementován takzvaný registrační mód, který lze dočasně povolit ve webovém rozhraní nadřazeného systému. Když je mód zapnutný, mohou se podřízené systémy registrovat pomocí k~tomu určeného API požadavku, a přístupový klíč jim je zaslán v~odpovědi.

Webové rozhraní nadřazeného systému bylo navrženo na základě předpokládaných případů užití, jako je například zobrazení monitorovaných garáží či událostí zaznamenaných v~konkrétní garáži. Přistup do rozhraní je ověřen pomocí hesla, které je uchováváno v~zabezpečené podobě pomocí \textit{hashovacího} algoritmu Argon2.

Jak je zmíněno výše, aplikace nadřazeného systému byla implementována v~jazyce Python s~pomocí frameworku Flask. Kromě toho bylo použito několik dalších nástrojů, jako například program pro ovládání GSM modulu Gammu nebo framework SQLAlchemy, který výrazně usnadnil práci s~databází systému.

% Nadřazený systém pracuje na základě modelu \textit{client/server}. Klienty jsou v tomto případě podřízené systémy přistupující k API a uživatel používající webové rozhraní. neco vo tim planovaci ale ne mmoc dlouhyho

K~výsledné aplikaci byla vytvořena sada automatizovaných testů, které ověřují funkčnost jednotlivých částí (například přihlašování do webového rozhraní). Tyto testy slouží především k~dalšímu rozšiřování aplikace, neboť s~jejich pomocí lze rychle ověřit, zda přidaný kód neporušil nějakou již implementovanou funkcionalitu. Pro další testování byl také vytvořen jednoduchý simulátor podřízeného systému.

V~rámci testování byla aplikace nasazena na virtuálním serveru služby Amazon EC2\footnote{Aplikace je do konce června 2018 dostupná na \url{https://demo-garaze.tk}.}. Účelem bylo vyzkoušet proces nasazování aplikace. Nasazenou aplikaci (včetně simulátoru) je také možné použít k~demonstraci funkce nadřazeného systému.

Výsledná aplikace nadřazeného systému byla poté úspěšně nasazena i na Raspberry Pi 3, čímž vzniklo zařízení, které bylo cílem této diplomové práce.

\subsection{Zdrojové kódy}

Při vývoji aplikace nadřazeného systému byl použit verzovací systém Git. Repozitář se zdrojovým kódem je volně k~dispozici (pod licencí LGPL) na serveru Github -- \url{https://github.com/ggljzr/mi-dip-impl}. Repozitář také obsahuje testy, předpřipravené konfigurační soubory a stručný návod k~nasazení aplikace na Raspberry Pi.

%\begin{itemize}
%    \item implementace
%    \begin{itemize}
%        \item spousta pouzitejch technologii/frameworku
%        \item z~tech zajimavejch flask, jinja, apscheduler
%    \end{itemize}
%\end{itemize}
\end{conclusion}

\bibliographystyle{templates/csn690}
\bibliography{mybibliographyfile}

\appendix

\chapter{Nasazení}

tady bude neco vo nasazovani na RPI (tj rozjet apache, vygenerovat certifikaty atd., viz \url{https://github.com/ggljzr/mi-dip-impl/tree/master/deployment})

tu databazi je potreba nastavit prava/dat nekam aby do ni ten apache moh zapisovat.

\label{sec:dp}
\chapter{Uživatelská příručka}
\label{sec:guide}

Tato uživatelská příručka se zabývá webovým rozhraním nadřazeného systému. Sekce příručky odpovídají stránkám jednotlivým stránkám rozhraní. V každé sekci jsou popsány možnosti, které má uživatel na dané stránce.

Přihlášení do webového rozhraní je prováděno pomocí hesla. Přihlašovací formulář je zobrazen na obrázku \ref{fig:login}.

\begin{figure}[h!]
    \centering
    \includegraphics[width=0.5\textwidth]{images/login.png}
    \caption[Přihlašovací formulář webového rozhraní]{Přihlašovací formulář webového rozhraní}
    \label{fig:login}
\end{figure}

\section{První přihlášení}

Pro první přihlášení je možné použít implicitní heslo \textit{password}. Poté je uživatel přesměrován na formulář se změnou hesla a vyzván ke změně z implicitní hodnoty (viz obrázek \ref{fig:password_change}).

\begin{figure}[h!]
    \centering
    \includegraphics[width=0.5\textwidth]{images/pwd_change.png}
    \caption[Výzva ke změně hesla]{Výzva ke změně hesla}
    \label{fig:password_change}
\end{figure}

\newpage

\section{Hlavní stránka}

\begin{figure}[h!]
    \centering
    \includegraphics[width=\textwidth]{images/mainpage.png}
    \caption[Hlavní stránka aplikace]{Hlavní stránka aplikace}
    \label{fig:mainpage}
\end{figure}

Po úspěšném přihlášení ze zobrazí hlavní stránka webového rozhraní. Zde má uživatel k dispozici následující možnosti (viz čísla na obrázku \ref{fig:mainpage}):

\begin{enumerate}
    \item \textbf{Nová garáž} -- vytvoření nové garáže pomocí uživatelského rozhraní. Nová garáž je přidána do seznamu garáží na této stránce.
    \item \textbf{Registrační mód} -- indikátor stavu registračního módu. Regstrační mód povoluje podřízeným systémům vytvářet nové garáže registračním požadavkem. Při registraci pomocí tohoto módu je podřízenému systému automaticky zaslán API klíč, a není nutné ho ručně nahrávat. Po zapnutní je registrační mód aktivní po dobu tří minut, poté se sám automaticky vypne.
    \item \textbf{Nastavení} -- uživatelské nastavení aplikace:
    \begin{itemize}
        \item \textbf{Změna hesla} -- možnost změny přístupového hesla do webového rozhraní.
        \item \textbf{Uživatel} -- viz sekci \ref{sec:guide_user_settings}.
        \item \textbf{Odhlásit} -- odhlášení z webového rozhraní.
    \end{itemize}
    \item Seznam sledovaných garáží (podřízených systémů):
    \begin{itemize}
        \item \textbf{ID} -- identifikátor garáže v systému, zároveň slouží jako odkaz na stránku garáže (viz sekci \ref{sec:guide_garage_page}).
        \item \textbf{Označení} -- Uživatelem zvolené označení garáže.
        \item \textbf{Dveře} -- Stav dveří garáže (otevřeno/zavřeno).
        \item \textbf{Poslední hlášení} -- Datum a čas posledního zaznamenaného hlášení.
        \item \textbf{Další plánované hlášení} -- Datum a čas dalšího plánovaného hlášení.
        \item \textbf{Stav} -- Stav garáže (OK, Nehlásí se, Detekce pohybu, Detekce kouře).
    \end{itemize}
    \item Záznam garáže.
\end{enumerate}

\newpage

\section{Stránka garáže}
\label{sec:guide_garage_page}

\begin{figure}[h!]
    \centering
    \includegraphics[width=\textwidth]{images/garage_page.png}
    \caption[Stránka garáže]{Stránka garáže}
    \label{fig:garage_page}
\end{figure}

Na obrázku \ref{fig:garage_page} je zobrazena stránka konkrétní garáže. Zde jsou k dispozici tyto možnosti:

\begin{enumerate}
    \item \textbf{Zneplatnit API klíč} -- vygeneruje nový API klíč garáže. Po jeho vygenerování již nebude podřízený systém v této garáži moci zasílat nové události. Pro obnovení přistupu je nutné nahrát na systém nový klíč, případně vytvořit novou garaž pomocí registračního módu.
    \item \textbf{Smazat garáž} -- smaže zobrazenou garáž, včetně zaznamenaných událostí.
    \item \textbf{Údaje o garáži}, včetně API klíče.
    \item \textbf{Nastavení garáže}:
    \begin{itemize}
        \item \textbf{Označení} -- uživatelem zvolené označení, které se zobrazí na hlavní stránce.
        \item \textbf{Perioda hlášení} -- perioda kontrolních hlášení podřízeného systému (v minutách).
        \item \textbf{Poznámka} -- volitelná poznánka ke garáži (max 256 znaků).
    \end{itemize}
    \item \textbf{Filtry událostí} -- možnost filtrovat zobrazené události podle typu.
    \item \textbf{Události} -- tabulka zaznamenaných událostí, včetně data a času.
\end{enumerate}


\section{Uživatelské nastavení}
\label{sec:guide_user_settings}

V uživatelském nastavení může uživatel měnit e-mailovou adresu, na kterou nadřazený systém zasílá upozornění o změně stavu garáže. Může zde také tato upozornění úplně zakázat.

\chapter{Seznam použitých zkratek}
\begin{description}
	\item[API] Application Programming Interface
    \item[AXI] Advanced eXtensible Interface
    \item[CPU] Central Processing Unit
    \item[CSRF] Cross-Site Request Forgery
    \item[DDR] Double Data Rrate
    \item[EEPROM] Electrically Erasable Programmable Read-Only Memory 
    \item[FPGA] Field-programmable gate array
    \item[GSM] Global System for Mobile communications
    \item[HDMI] High-Definition Multimedia Interface
    \item[HTML] HyperText Markup Language 
	\item[HTTP] HyperText Transfer Protocol 
    \item[HTTPS] HTTP Secure
    \item[I2C] Inter-Integrated Circuit
    \item[IP] Intellectual Property
    \item[JSON] JavaScript Object Notation
    \item[LCD] Liquid-Crystal Display
    \item[LPDDR] Low Power DDR
    \item[LTS] Long Term Support
    \item[MQTT] Message Queuing Telemetry Transport
    \item[MVC] Model-View-Controller
    \item[NTP] Network Time Protocol
    \item[OS] Operační systém
    \item[OSI] Open Systems Interconnection
    \item[PSK] Pre-Shared Key
    \item[QoS] Quality of Service
    \item[RAM] Random Access Memory
    \item[RTC] Real Time Clock
    \item[SD] Secure Digital
    \item[SIM] Subscriber Identity Module
    \item[SMTP] Simple Mail Transfer Protocol
    \item[SSD] Solid State Drive
    \item[SSL] Secure Socket Layer
    \item[SoC] System on Chip
    \item[TCP/IP] Transmission Control Protocol / Internet Protocol
    \item[TLS] Transport Layer Security
    \item[URL] Uniform Resource Locator
    \item[USB] Universal Serial Bus
    \item[UUID] Universally Unique Identifier
    \item[WSGI] Web Server Gateway Interface
\end{description}

\chapter{Obsah přiloženého CD}

\begin{figure}
	\dirtree{%
		.1 readme.txt\DTcomment{stručný popis obsahu CD}.
        .1 mi-dip\DTcomment{Git repozitář se zdrojovým kódem práce ve formátu \LaTeX{}}.
        .1 mi-dip-impl\DTcomment{Git repozitář se zdrojovým kódem implementace (včetně testů)}.
        .1 DP\_Cervenka\_Ondrej\_2018.pdf\DTcomment{text práce ve formátu PDF}.
	}
\end{figure}

\end{document}
