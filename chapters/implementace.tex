\chapter{Implementace}
\label{sec:im}

V~této kapitole je popsána implementace nadřazeného systému podle návrhu z~kapitoly \ref{sec:de}. Jednotlivé sekce tedy popisují implementační detaily dílčích částí návrhového vzoru MVC. Poslední sekce \ref{sec:im_auth} se zabývá implementací autentizace uživatele.

\section{Struktura aplikace}

Aplikace nadřazeného systému je rozdělena do tří modulů:

\begin{itemize}
    \item \texttt{mod\_main} -- hlavní modul aplikace. V~tomto modulu je implementován \textit{model} systému popsaný v~sekci \ref{sec:de_model}. \textit{Model} je využíván i ostatními moduly (konkrétně modulem \texttt{mod\_api}). Dále tento modul implementuje webové rozhraní správy nadřazeného systému.
    \item \texttt{mod\_api} -- modul implementující API systému. Zde je implementována část \textit{controlleru} ze sekce \ref{sec:de_api}, definující URL, pomocí kterých mohou podřízené systémy komunikovat s~nadřazeným systémem.
    \item \texttt{mod\_auth} -- modul implementující autentizaci uživatele při přihlašování do webového rozhraní.
\end{itemize}

K~rozdělení aplikace na jednotlivé moduly je použit nástroj frameworku Flask nazvaný \textit{blueprints} \cite{flask_blueprints}.  Hlavní důvod k~využití modulů je oddělení částí aplikace podle jejich funkce.

Při strukturování aplikace jsem vycházel z~článku \textit{How To Structure Large Flask Applications} \cite{flask_large}.

\section{Implementace \textit{modelu}}

Jak je zmíněno v~sekci \ref{sec:de_model}, \textit{model} aplikace je tvořen třídami \texttt{Garage} a \texttt{Event}. Tyto třídy přímo využívají databázi nadřazeného systému, k~jejich implementaci je proto použit framework SQLAlchemy \cite{sqlalchemy}, který výrazně usnadní práci s~databází.

Tento framework poskytuje přístup k~SQL databázím přímo z~jazyka Python, takže není nutné psát prakticky žádný SQL kód. Tabulku v~databázi je možné definovat jako Python třídu s~přísušnými atributy a SQLAlchemy vytvoří odpovídající databázové schéma. 

SQLAlchemy také umožňuje snadno definovat databázové vztahy. V~\textit{modelu} nadřazeného systému se vyskytuje pouze vztah $1:N$ mezi třídami \texttt{Garage} a \texttt{Event} (jedna garáž má mnoho událostí, každá událost má právě jednu garáž). Tento vztah je definován následujícím způsobem:

\begin{listing}[htbp]
\caption{\label{lst:db_relationship} Vytvoření vztahu $1:N$ mezi třídami \texttt{Garage} a \texttt{Event}}
\begin{minted}[bgcolor=codebg]{python}
from sqlalchemy import Column, ForeignKey, Integer
from sqlalchemy.ext.declarative import declarative_base
from sqlalchemy.orm import relationship

Base = declarative_base()

# třídy Event a Garage
# dědí SQLAlchemy metody
# třídy Base
class Event(Base):
    ...
    # odkaz na příslušnou garáž
    garage_id = Column(Integer, ForeignKey(
        'Garage.id'), nullable=False)

class Garage(Base): 
    ...
    # definice 1:N vztahu mezi garáží a událostí
    events = relationship('Event', backref='Garage')
\end{minted}
\end{listing}

\subsection{Kontrola zmeškaných hlášení}
\label{sec:im_scheduler}

V~databázi nadřazeného systému je potřeba pravidelně provádět kontrolu, zda podřízené systémy zaslaly v~očekávaný čas kontrolní hlášení. Pokud bylo plánované hlášení promeškáno, je nutné změnit stav příslušné garáže na \uv{Nehlásí se}.

Aplikace nadřazeného systému nemá v~zásadě možnost, jak tuto kontrolu sama iniciovat, neboť pouze reaguje na příchozí požadavky (od uživatele či podřízeného systému). Provedení kontroly může být důsledkem takového požadavku, například pokud uživatel otevře hlavní stránku webového rozhraní. 

Provádět kontrolu hlášení pouze v~reakci na vnější podnět však není dostatečné. Pokud by nadřazený systém musel pro provedení kontroly hlášení čekat interakci s~webovým rozhraním (nebo například s~podřízeným systémem), nemuselo by vůbec dojít ke změně stavu garáže a tedy ani k~odeslání notifikačního e-mailu. Je tedy nutné zajistit pravidelné provádění kontrol na základě vnitřního podnětu.

Tento problém jsem vyřešil použítím plánovače APScheduler \cite{apscheduler}. APScheduler funguje jako knihovna do Pythonu, a umožňuje plánovat provádění zvolených funkcí. Nejde tedy o~externí program, plánovač je přímo součástí kódu nadřazeného systému \cite{apscheduler}. Vytvoření pravidelné kontroly hlášení podřízených systémů je implementováno tímto způsobem:

\begin{listing}[htbp]
\caption{\label{lst:scheduler_check} Pravidelná kontrola hlášení podřízených systémů pomocí knihovny APScheduler. Po startu plánovače je každých 5 minut volána metoda \texttt{Garage.check\_reports()}}
\begin{minted}[bgcolor=codebg]{python}

from apscheduler.schedulers.background import \
    BackgroundScheduler

# BackgroundScheduler běží v~samostatném vlákně,
# neblokuje tedy webovou aplikaci
scheduler = BackgroundScheduler()

# přidání pravidelného úkolu do plánovače
# Garage.check_reports je statická metoda
# třídy Garage, která provede kontrolu hlášení
# u~všech garáží v~databázi (a případně upraví jejich stav)
scheduler.add_job(Garage.check_reports, 'interval', minutes=5)
scheduler.start()
\end{minted}
\end{listing}

Plánovač se kromě kontroly hlášení hodí také při vypínání registračního módu. Ten z~bezpečnostních důvodů po aktivaci běží po dobu tří minut. Jeho vypnutí je naplánováno obdobně jako kontrola hlášení, jediný rozdíl je, že úkol není spouštěn v~pravidelném intervalu, ale pouze jednou.

\subsection{Zasíláním upozornění}

%posilani mailu pomoci toho eventu zmeny v sqlalchemy

%posilani mailu obecne jak je implementovany

\section{Implementace \textit{view}}

Tato část se věnuje implementaci stránek webového rozhraní nadřazenéh systému. Jednotlivé HTML stránky rozhraní jsou generovány pomocí šablonovacího systému Jinja \cite{jinja}, který je součástí frameworku Flask \cite{flask_templates}. 

Jinja umožňuje dynamické generování HTML stránek z~předem vytvořených šablon a dalších vstupnů, v~tomto případě především dat z~\textit{modelu} aplikace. Tato data jsou šabloně předána \textit{controllerem} (viz sekci \ref{sec:im_controller}). V~šablonách Jinja používá syntaxi založenou na Pythonu \cite{jinja}. Například HTML šablona pro tabulku se zaznamenanými událostmi vypadá následovně:

\begin{listing}[htbp]
\caption{\label{lst:jinja_table} HTML šablona tabulky zaznamenaných událostí, využívající šablonovací systém Jinja. Proměnná \texttt{garage} je šabloně předána \textit{controllerem} aplikace}
\begin{minted}[bgcolor=codebg]{html}
<table class=basic_table>
    <tr>
    <th>Seznam událostí</th>
    </tr>
    <!-- použítí cyklu v~systému Jinja -->
    
        <tr>
            <!-- přístup k~proměnné pomocí bloku {{ }} -->
            <td class=event_type_{{event.type}}>
                {{ event | event_filter }}
            </td>
        </tr>
     <!--konec bloku cyklu -->
</table>
\end{minted}
\end{listing}

Jinja také ďovoluje definici filtrů, sloužících ke zpracování zobrazovaných dat \cite{jinja}. V~ukázce \ref{lst:jinja_table} je použit filtr \texttt{event\_filter}. Ten dostane jako vstup instanci třídy \texttt{event} a vytvoří její textovou reprezentaci s~údaji o~typu a času události. Tím je oddělena implementace této třídy v~\textit{modelu} a její zobrazení ve \textit{view}.

\subsection{Formuláře}
\label{sec:im_forms}

Webové rozhraní obsahuje několik formulářů, které slouží například k~editaci garáže či změně hesla. Tyto formuláře jsou implementovány pomocí frameworku WTForms, který je opět integrován ve Flasku \cite{flask_wtf}.

Framework poskytuje třídu \texttt{FlaskForm}, pomocí které je možné vytvořit podtřídy popisující jednotlivé formuláře \cite{flask_wtf}. V~těchto podtřídách jsou pak definovány pole formuláře.

U~některých polí je potřeba provádět kontrolu zadávaných hodnot (například dodržení minimální délky hesla). To je provedeno pomocí další součásti WTForms, nazvané validátory \cite{flask_wtf}. Pomocí těchto předdefinovaných validátorů lze na vstupní data aplikovat nejrůznější omezení (délka řetězce, rozsah číselných hodnot atd.) a určit chybovou zprávu zobrazenou při jejich porušení \cite{flask_wtf}.

Využití tohoto frameworku při implementaci formuláře pro editaci údajů garáže popisuje ukázka \ref{lst:wtform}. Vygenerování HTML kódu formuláře, včetně zobrazení chybových zpráv po odeslání neplatného formuláře, je provedeno pomocí systému Jinja.

\begin{listing}[htbp]
\caption{\label{lst:wtform} Implementace formuláře pro editaci údajů garáže pomocí frameworku WTForms. Při kontrole vstupu je ověřen rozsah zadávané periody}
\begin{minted}[bgcolor=codebg]{python}
from flask_wtf import FlaskForm
from wtforms import StringField, IntegerField, TextAreaField
from wtforms.validators import NumberRange

class GarageForm(FlaskForm):
    tag = StringField('Označení')
    period = IntegerField('Perioda hlášení (minuty)', 
        validators=[
        NumberRange(1, 999, 
            message='Perioda musí být mezi 1 a 999')
        ])
    note = TextAreaField('Poznámka')
\end{minted}
\end{listing}

\subsection{Filtrování událostí}

Při zobrazení seznamu událostí konkrétní garáže je možné tyto události filtrovat podle jejich typu. Filtrování je prováděno na straně klienta (v~prohlížeči) pomocí Javascriptu a knihovny jQuery \cite{jquery_about}.

\section{Implementace \textit{controlleru}}
\label{sec:im_controller}

\textit{Controller} aplikace je implementován spárováním URL a funkce, která se má provést při příchozím požadavku na toto URL. Jako odpověď na požadavek je pak klientovi zaslána návratová hodnota funkce. Tento princip (demonstrovaný v~ukázce \ref{lst:route}) představuje základní stavební kámen při tvorbě \textit{controlleru} ve frameworku Flask.

Pro spárování funkce a URL je použit dekorátor \texttt{route}. Dekorátor je konstrukt jazyka Python, který umožňuje rozšířit danou funkci či metodu o~další funkcionalitu \cite{python_decorators}. Pomocí dekorátoru \texttt{route} je definováno URL a případně i povolené HTTP metody, které může klient při požadavku použít. Implicitně je povolena pouze metoda \texttt{get} \cite{flask_api}. 

\begin{listing}[htbp]
\caption{\label{lst:route} Přiřazení funkce k~URL. Funkce \texttt{index} bude zavolána při každém příchozím požadavku s~metodou \texttt{get} na kořenové URL \texttt{/}. Návratová hodnota funkce je vygenerovaná HTML stránka, která bude zaslána klientovi}
\begin{minted}[bgcolor=codebg]{python}
from flask import Blueprint, render_template

from .models.garage import Garage 

# deklarace blueprintu mod_main
mod_main = Blueprint('main', __name__)

# URL a příslušná akce jsou definovány
# vzhledem k~blueprintu
@mod_main.route('/')
@login_required # kontrola přihlášení uživatele
def index():
    # získání údajů o~všech garážích
    garages = Garage.query.all()

    # vygenerování stránky pomocí Jinja
    # zobrazovaná data jsou předána jako
    # parametry funkce render_template()
    return render_template('main/index.html', 
                            garages=garages, 
                            reg_mode=Garage.reg_mode)
\end{minted}
\end{listing}

Pokud funkce zpracovávající požadavek mění obsah databáze (například vytváří novou garáž), je u~požadavku vyžadována metoda \texttt{post}. Tyto požadavky jsou také chráněny proti CSRF útoku. Do každého formuláře používajícího metodu \texttt{post} je vložen náhodně generovaný \textit{token}, podle kterého může aplikace ověřit původ požadavku. Ochrana proti CSRF je automaticky zapnuta u~všech formulářů vytvořených pomocí WTForms (viz sekci \ref{sec:im_forms}) \cite{flask_wtf}.

%csrf, viz  napsat ze pro add garage a revoke key misto normalni routy, getu a linku pouzivame formulare a post kvuli vochrane pred csrf, viz \url{https://stackoverflow.com/questions/6812765/how-to-demonstrate-a-csrf-attack}. Timhle utokem by nekdo moh vytvaret garaze a rusit api klice, coz neni naka velka skoda ale spis na votravovani no (u~vostatnich veci (tj hlavne change password) to bylo uz driv v~pohode protoze byly pouzity ty flaskforms)

\subsection{Implementace API}

%vyjimka z csrf (neni potreba login, prokazuje se klicem)

Způsobem popsaným v~sekci \ref{sec:im_controller} je implementováno i API nadřazeného systému. Operace a URL ze sekce \ref{sec:de_api} jsou spárovány s~příslušnými funkcemi, upravujícími \textit{model} aplikace.

Jelikož se podřízené systémy prokazují pomocí API klíče, není zde nutné provádět žádnou kontrolu přihlášení jako při zpracování požadavků ve webovém rozhraní. Pouze je zkontrolována přítomnost zaslaného klíče v~databázi. Také není nutné implementovat ochranu proti CSRF útokům.

\section{Autentizace a přihlášení uživatele}
\label{sec:im_auth}

Autentizace uživatele je provedena zadáním hesla. Heslo je zadáváno pomocí formuláře na přihlašovací stránce a posíláno požadavkem \texttt{post}. Použití metody \texttt{get} není vhodné, neboť heslo jako součást URL zůstane uchované jednak v~historii prohlížeče, jednak v~záznamu požadavků na serveru.

Pokud uživatel poskytl správné heslo, je v~jeho HTTP relaci nastavena proměnná \texttt{logged\_in} na hodnotu \texttt{true}. Tato proměnná slouží ke kontrole přihlášení uživatele při přístupu k~dalším částem webového rozhraní. Kód provádějící kontrolu je vkládán do funkcí pomocí dekorátoru \texttt{login\_required}, jehož použití lze vidět v~ukázce \ref{lst:route} (bližší informace o~dekorátorech je možno najít v~sekci \ref{sec:im_controller}). Odhlášení je provedeno nastavením hodnoty \texttt{logged\_in} na \texttt{false}.

%ukladani hesla (asi jen ze pouzivame bcrypt a ze to generuje i sul kterou to pripoji k tomu hashi hned takze se nemusi nic resit)

%v~testovani je mozny tenhle utok demonstrovat a ukazat jak pekne nam to funguje (voproti ty prvni verzi kde byl na add garage normalne get). Ted se pri tim pokusu vo utok normalne zvobrazi invalid csrf token nebo tak neco. Pro porovnani zmen puvodni a vopraveny verze viz commit e52a54b2caa8eead85e8df28c738356a7541a1c4 (password redirect, to je posledni verze s~timhle bugem)
