\chapter{Testování}
\label{sec:te}

K~aplikaci jsem vytvořil sadu automatizovaných testů, které ověřují funkčnost jednotlivých komponent. Konkrétně jde o~testy pro \textit{model} aplikace (zde je testována především třída \texttt{garage}), webové rozhraní a API systému. Tyto testy byly implementovány pomocí frameworku pytest \cite{pytest}.

Části aplikace, u~kterých je automatizované testování problematikcé, jako například funkce využívající APScheduler (viz sekce \ref{sec:im_scheduler}) nebo zasílání emailů, jsem otestoval ručně.

\section{Testovací konfigurace}

%napsat ze sem musel vypnout csrf vochranu kvuli snadnejsimu testovani, a test jestli ta csrf vochrana funguje je v jiny kapitole

\section{Testování \textit{modelu}}

Testování \textit{modelu} aplikace spočívalo především v~testování metod třídy \textit{garage}. Třída \textit{event} slouží pouze k~uchování dat o~události a neimplementuje žádné metody, které by bylo možné testovat.

Testovány byly následující funkce:

\begin{itemize}
    \item Přidání garáže.
    \item Smazání garáže.
    \item Kontrola zmeškaných hlášení.
    \item Zaslání kontrolího hlášení.
    \item Změna stavu po zaslání události.
    \item Zneplatnění API klíče.
\end{itemize}

Zde stojí za zmínku test kontroly zmeškaných hlášení. V~něm je garáži zasláno kontrolní hlášení, čímž je určen čas dalšího očekávaného hlášení. Po překročení tohoto času je potřeba otestovat, zda byl stav garáže změnen na \uv{Nehlásí se}.

Jelikož je minimální perioda hlášení 1 minuta, prosté čekání (například pomocí funkce \texttt{sleep()}) by proces testování neúnosně zpomalilo (pro srovnání spuštění všech testů aplikace na běžném počítači zabere asi 4 sekundy). Vhodnější je tedy časový posun nasimulovat.

K~tomu jsem použil knihovnu freezegun \cite{freezegun}.

\section{Testování API}

\section{Testování webového rozhraní}

\section{Testování autentizace uživatele}

\subsection{Testování ochrany proti CSRF}

% test je udelanej v test_auth

\section{Test HTTPS konfigurace}

% lol nasadime to na aws nebo digital ocean, registrujem domenu, dame lets encrypt cert a zkusime sslabs test

% v tyhle kapitole taky poresit to ze se jako bottleneck ty komunikace zda bejt ten https handshake a ze by se to asi dalo vyresti pouzitim sessions na strane klienta http://docs.python-requests.org/en/latest/user/advanced/#session-objects