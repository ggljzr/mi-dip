Cílem této práce je vytvořit nadřazený systém pro monitorování garážového komplexu. Výsledná aplikace bude komunikovat pomocí WiFi či Ethernetu s podřízenými systémy (zasílajícími údaje z čidel v garážích). Na základě získaných dat pak bude udržován stav jednotlivých garáží a vytvářena historii událostí.

Systém bude poskytovat webového rozhraní pro administraci. V tom bude možné přidávat a odebírat podřízené systémy, zobrazovat jejich stav a zaznamenané události.

Vzhledem k povaze zadání je nutné systém navrhnout s ohledem na zabezpečení přenášených informací před odposloucháváním či manipulací. Též je nutné autorizovat uživatele přistupjící do webového rozhraní.

Dalším důležitým požadavkem je snadná rozšiřitelnost o nové funkce. Systém by mělo být možné v budoucnu doplnit o možnost správy rozdílných podřízených systémů (například subsystémy pro sledování skladových zásob) či integraci s mobilní aplikací. Bude tedy potřeba navrhnout vhodné API pro předávání informací mezi systémem a jeho klienty. 

V této práci se chci zaměřit na tvorbu aplikace na jedné konkrétní platformě, jako například \textit{Raspberry Pi}. Aplikace spoulu se zvolenou platformou by pak měla tvořit kompletní zařízení, které bude možné po základní konfiguraci (připojení do WiFi sítě, nastavení hesla) hned nasadit.

Výsledné řešení by však mělo být dostatečne nezávislé na zvolené platformě. Tudíž by neměl být problém spustit systém například na osobním počítači či virtuálním serveru.


