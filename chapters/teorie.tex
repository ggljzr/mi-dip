\chapter{Teoretický základ}
\label{sec:te}

\section{Základní pojmy}

do ty teorie dat spis veci jako protokoly, koncepty, nebo obecny technologie

k softwaru dat citaci primo v textu

\begin{itemize}
    \item uzivatel -- use case diagramy, nebo ty mozna az v navrhu
    \item klic
    \item api -- to uz ale je ve zkratkach, jestli to tady budem nak rozepisovat tak popsat taky json
    \item client/server
    \item tcp/ip
    \item osi
    \item http/https, pozadavek, metody, navartovy kody, hlavicky pozadavku atd, relace (session)
    \item hash, u hashe zminit bcrypt a eventuelne jiny bezpecny algoritmy, sul
    \item mqtt
    \item certifikát, k tomu veřejný a soukromý klíč, self-signed jen treba lehce zminit, protoze je uz dost popsanej v ty http sekci
    \item vícevláknová obsluha
    \item nadrazeny system
    \item podrizeny system
    \item mitm attack
    \item csrf
    \item publisher/subscriber
    \item webové rozhraní
    \item unix time
    \item cloud
    \item attack surface
    \item relacni databaze, databazovy schema, vztahy (1:n, atd)
    \item navrhovy vzory -- mvc
\end{itemize}

\section{\textit{Hashování}}

\textit{Hashováním} se obecně rozumí vytvoření otistku pevné délky -- \textit{hashe} -- z libovolných vstupních dat. Důležitou vlastností je jednosměrnost tohoto procesu, tedy že z vytvořeného otisku již nelze zrekonstruovat původní vstup. \cite{hash_crackstation}

V této práci je \textit{hashování} použito při ukládání hesel. Heslo uložené v čitelné podobě není při úniku souboru s heslem nijak chráněno. Proto je vhodné uložit místo hesla samotného jeho \textit{hash}. Pokud dojde k úniku \textit{hashovaných} hesel, je pro útočníka velmi problematické  získat z otisků čitelná hesla \cite{hash_crackstation}.

\textit{Hashování} lze provést pomocí mnoha algortimů, ne všechny se však hodí k zabezpečení ukládaných hesel. Některé algoritmy, jako například MD5, nejsou dostatečně výpočetně náročné, takže umožňují efektivní útoky hrubou silou (lze dostatečně rychle spočítat všechny možné kombinace vstupů do dané délky a tím zjistit jaký vstup odpovídá danému \textit{hashi}) \cite{hash_crackstation}. 

Pro bezpečné ukládání hesel je tedy nutné použít dostatečné náročný algoritmus. V této práci je použit algoritmus Bcrypt, který je považován za vhodný k \textit{hashování} uložených hesel \cite{hash_crackstation}.

\subsection{Solení \textit{hashů}}

\textit{Hashování} hesla je vhodné doplnit o proces takzvaného \uv{solení}. Při něm se nevytváří otisk samotného hesla, ale hesla doplněného o náhodně vygenerovaný řetězec -- sůl. To ztěžuje použítí předpočítaných tabulek (například tabulky s otisky často používaných hesel) pro útoky hrubou silou \cite{hash_crackstation}.

Pro bližší informace o bezpečném ukládání hesel doporučuji článek \textit{Salted Password Hashing - Doing it Right} \cite{hash_crackstation}.
