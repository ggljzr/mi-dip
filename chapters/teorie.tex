\chapter{Teoretický základ}
\label{sec:te}

\section{Základní pojmy}

do ty teorie dat spis veci jako protokoly, koncepty, nebo obecny technologie

k softwaru dat citaci primo v textu

\begin{itemize}
    \item uzivatel -- use case diagramy, nebo ty mozna az v navrhu
    \item klic
    \item api -- to uz ale je ve zkratkach, jestli to tady budem nak rozepisovat tak popsat taky json
    \item client/server
    \item tcp/ip
    \item osi
    \item http/https, pozadavek, metody, navartovy kody, hlavicky pozadavku atd, relace (session)
    \item hash, u hashe zminit bcrypt a eventuelne jiny bezpecny algoritmy, sul
    \item mqtt
    \item certifikát, k tomu veřejný a soukromý klíč, self-signed jen treba lehce zminit, protoze je uz dost popsanej v ty http sekci
    \item vícevláknová obsluha
    \item nadrazeny system
    \item podrizeny system
    \item mitm attack
    \item csrf
    \item publisher/subscriber
    \item webové rozhraní
    \item unix time
    \item cloud
    \item attack surface
    \item relacni databaze, databazovy schema, vztahy (1:n, atd)
    \item navrhovy vzory -- mvc
\end{itemize}

\section{Bezpečnost}
