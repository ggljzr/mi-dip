co sem udelal

\begin{itemize}
    \item analyza
    \begin{itemize}
    \item definice nadrazeneho systemu, definice pozadavku na nej (specifikace), popis podrizeneho systemu, definicie udalosti jako zakladni komunikacni jednotky, bezpecnost a spolehlivost systemu
    \item vyber kom. protokolu, vyhody a nevyody https, mqtt, volba certifikatu (delame prototyp), dalsi moznosti provozu (cloud), komunikacni modely client/server, publisher/subscriber
    \item autentizace podrizenejch systemu, api klice
    \item problem sireni api klicu na subsystemy -- reg mod
    \item vyber dalsiho softwaru, programovaciho jazyku, frameworku (pochvalit framework flask, pohodova implementace)
    \item vyber uloziste (db vs logy, presistence), resil sem: systemove naroky, rust velikosti
    \item notifikace, vyber zpusobu zasilani -- co sme vsechno vyzkouseli (gmail, smsky, gsm twilio)
    \item vyber platformy, podporovany sw, rpi vs zybo, zybo zajimay ale moc drahy, zminit rtc
    \end{itemize}
    \item navrh
    \begin{itemize}
        \item navrh systemu, pouziti mvc, moznosti dalsich controlleru (mqtt) -- rozsiritelnost
        \item stav garazi, fsm, prechody stavu, notifikace pri zmene stavu -- co to resi za problemy (casty maily), jak resit prechod zpet do stavu OK napr po pozaru
        \item api systemu, navrh, pridavani udalosti, registracni mod
        \item uzivatelsky rozhrani, prihlasovani, use casy, prvni prihlaseni (defaultni heslo), uzivatelsky nastaveni
        \item jak sem resil problem s~autorizaci pristupu k~posilani mailu
    \end{itemize}
    \item implementace
    \begin{itemize}
        \item spousta pouzitejch technologii/frameworku
        \item z~tech zajimavejch flask, jinja, apscheduler
    \end{itemize}
    \item testovani
    \begin{itemize}
        \item k~aplikaci je sada unit testu
        \item pri testovani bylo potreba vyresit: CSRF, posun casu, prihlaseni pomoci modifikace session -- tady pomoh flaskc
        \item behem testovani se aplikaci po mirnym natlaku podarilo nasadit na virtualnim serveru, s~domenou a https certifikatem (pravym). tohle bylo uzitecny protoze to vodhalilo par chybek
        \item byl vytvoren jednoduchy simulator podrizeneho systemu
    \end{itemize}
    \item sepsani uzivatelsky prirucky
    \item napsat ze zdrojovy kody, testy a strucnej navod k~nasazeni aplikace je na githubu a ze sme pouzivali github celou dobu
\end{itemize}