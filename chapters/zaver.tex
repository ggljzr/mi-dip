
V~rámci práce byl vytvořen nadřazený systém pro zabezpečení garáží. Systém má za úkol sbírat zaznamenaná dat od podřízených systémů monitorujících jednotlivé garáže a v~případě nebezpečí odeslat zprávu s~upozorněním. Uživatel (majitel či správce garážového komplexu) může systém spravovat pomocí webového rozhraní.

Aplikace nadřazeného systému byla vytvořena v~jazyce Python a nasazena na jednodeskovém počítači Raspberry Pi 3. Výsledkem práce je tedy jednoúčelové zařízení, které je po krátké konfiguraci možné hned použít.

Text práce lze rozdělit do pěti částí: teoretický úvod, analýza, návrh, implementace a testování. Teoretický úvod slouží k~definici použitého názvosloví. Kromě toho obsahuje stručný popis některých pojmů a konceptů, které se v~práci vyskytují.

V~analýze byla na základě zadání práce vytvořena specifikace nadřazeného systému, popisující jeho základní chování a požadavky, které má splňovat. Podle této specifikace byl pak zvolen software a hardware vhodný k~implementaci.

Významný prostor byl věnován volbě protokolu, na kterém bude postavena komunikace mezi nadřazeným a podřízenými systémy. Analýza se blíže zaměřila na protokoly HTTPS a MQTT. Kromě základních funkcí obou protokolů byly zkoumány také možnosti zabezpečení a autentizace účastníků komunikace.

Kromě protokolu byl v~analýze také zvolen způsob ukládání dat (databázový systém SQLite3) a odesílání upozornění. Právě zasílání upozornění se ukázalo jako poměrně zajímavý problém s~několika možnými řešeními. V~rámci analýzi byly otestovány internetové služby pro zasílání e-mailů (Gmail) a SMS (Twilio). Kromě toho bylo také otestováno zasílání SMS pomocí GSM modulu připojeného přes USB port. Tento způsob byl nakonec použit ve výsledném zařízení.

Pro implementaci aplikace nadřazeného systému byl vybrán jazyk Python a webový framework Flask. Hlavní důvod této volby byla moje předchozí zkušenost s~těmito nástroji.

V~závěru analytické části byla zkoumána vhodná hardwarová platforma pro tvorbu výsledného zařízení. Zkoumány byly jednodeskové počítače Raspberry Pi 3 a Zybo Zynq-7000. Deska Zynq-7000 byla zajímavá především díki integraci FPGA obvodu, ukázalo se však, že ten by měl v~této práci pouze velmi omezené využití. Jako platforma pro tvorbu zařízení bylo tedy zvoleno Raspberry Pi 3, které poskytuje vyšší procesorový výkon za výrazně nižší cenu.

Na základě této analýzi byl vytvořen návrh aplikace, kde byl jako základ použit vzor \textit{model-view-controller}.

\begin{itemize}
    \item navrh
    \begin{itemize}
        \item navrh systemu, pouziti mvc, moznosti dalsich controlleru (mqtt) -- rozsiritelnost
        \item stav garazi, fsm, prechody stavu, notifikace pri zmene stavu -- co to resi za problemy (casty maily), jak resit prechod zpet do stavu OK napr po pozaru
        \item api systemu, navrh, pridavani udalosti, registracni mod
        \item uzivatelsky rozhrani, prihlasovani, use casy, prvni prihlaseni (defaultni heslo), uzivatelsky nastaveni
        \item jak sem resil problem s~autorizaci pristupu k~posilani mailu
    \end{itemize}
    \item implementace
    \begin{itemize}
        \item spousta pouzitejch technologii/frameworku
        \item z~tech zajimavejch flask, jinja, apscheduler
    \end{itemize}
    \item testovani
    \begin{itemize}
        \item k~aplikaci je sada unit testu
        \item pri testovani bylo potreba vyresit: CSRF, posun casu, prihlaseni pomoci modifikace session -- tady pomoh flaskc
        \item behem testovani se aplikaci po mirnym natlaku podarilo nasadit na virtualnim serveru, s~domenou a https certifikatem (pravym). tohle bylo uzitecny protoze to vodhalilo par chybek
        \item byl vytvoren jednoduchy simulator podrizeneho systemu
    \end{itemize}
    \item sepsani uzivatelsky prirucky
    \item napsat ze zdrojovy kody, testy a strucnej navod k~nasazeni aplikace je na githubu a ze sme pouzivali github celou dobu
\end{itemize}