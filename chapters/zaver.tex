
V~rámci práce byl vytvořen nadřazený systém pro zabezpečení garáží. Systém má za úkol sbírat zaznamenaná dat od podřízených systémů monitorujících jednotlivé garáže a v~případě nebezpečí odeslat zprávu s~upozorněním. Uživatel (majitel či správce garážového komplexu) může systém spravovat pomocí webového rozhraní.

Aplikace nadřazeného systému byla vytvořena v~jazyce Python a nasazena na jednodeskovém počítači Raspberry Pi 3. Výsledkem práce je tedy jednoúčelové zařízení, které je po krátké konfiguraci možné hned použít.

Text práce lze rozdělit do pěti kapitol: teoretický úvod, analýza, návrh, implementace a testování. Teoretický úvod slouží k~definici použitého názvosloví. Kromě toho obsahuje stručný popis některých pojmů a konceptů, které se v~práci vyskytují.

V~analýze byla na základě zadání práce vytvořena specifikace nadřazeného systému, popisující jeho základní chování a požadavky, které má splňovat. Podle této specifikace byl pak zvolen software a hardware vhodný k~implementaci.

Významný prostor byl věnován volbě protokolu, na kterém bude postavena komunikace mezi nadřazeným a podřízenými systémy. Analýza se blíže zaměřila na protokoly HTTPS a MQTT. Kromě základních funkcí obou protokolů byly zkoumány také možnosti zabezpečení a autentizace účastníků komunikace. Pro komunikaci s~podřízenými systémy byl nakonec zvolen protokol HTTPS.

Kromě protokolu byl v~analýze také zvolen způsob ukládání dat (databázový systém SQLite3) a odesílání upozornění. Právě zasílání upozornění se ukázalo jako poměrně zajímavý problém s~několika možnými řešeními. V~rámci analýzi byly otestovány internetové služby pro zasílání e-mailů (Gmail) a SMS (Twilio). Kromě toho bylo také otestováno zasílání SMS pomocí GSM modulu připojeného přes USB port. Tento způsob byl nakonec použit ve výsledném zařízení.

Pro implementaci aplikace nadřazeného systému byl vybrán jazyk Python a webový framework Flask. Hlavní důvod této volby byla moje předchozí zkušenost s~těmito nástroji.

V~závěru analytické části byla zvolena vhodná hardwarová platforma pro tvorbu výsledného zařízení. Zkoumány byly jednodeskové počítače Raspberry Pi 3 a Zybo Zynq-7000. Deska Zynq-7000 byla zajímavá především díki integraci FPGA obvodu, ukázalo se však, že ten by měl v~této práci pouze velmi omezené využití. Jako platforma pro tvorbu zařízení bylo tedy zvoleno Raspberry Pi 3, které poskytuje vyšší procesorový výkon za výrazně nižší cenu.

Na základě této analýzi byl vytvořen návrh aplikace nadřazeného systému, kde byl jako základ použit vzor \textit{model-view-controller}. Díky tomuto vzoru by měl být systém poměrně snadno rozšiřitelný. Například přidání možnosti komunikovat s~podřízenými systémy pomocí protokolu MQTT lze realizovat pouze doplněním vhodného \textit{controlleru}.

Dále byl v~kapitole návrh popsán způsob vyhodnocování stavu monitorovaných garáží a odesílání upozornění.

Pro komunikaci s~podřízenými sytémy bylo navrženo API, pomocí kterého mohou nadřazenému systému zasílat události popisující stav garáže (například při detekci kouře). Podřízené systémy se při přístupu k~API prokazují pomocí náhodně generovaných klíčů.

Zde bylo potřeba vyřešit problém distribuce klíčů podřízeným systémům. K~tomu byl implementován takzvaný registrační mód, který lze dočasně povolit ve webovém rozhraní nadřazeného systému. Když je mód zapnutný, mohou se podřízené systémy registrovat pomocí k~tomu určenému API požadavku, a přístupový klíč jim je zaslán v~odpovědi.

Webové rozhraní nadřazeného systému bylo navrženo na základě předpokládaných případů užití, jako je například zobrazení monitorovaných garáží či událostí zaznamenaných v~konkrétní garáži. Přistup do rozhraní je zabezpečen heslem, které je uchováváno v~zabezpečené podobě pomocí \textit{hashovacího} algoritmu Argon2.

Jak je zmíněno výše, aplikace nadřazeného systému byla implementována v~jazyce Python s~pomocí frameworku Flask. Kromě toho bylo použito několik dalších nástrojů, jako například program pro ovládání GSM modulu Gammu nebo framework SQLAlchemy, který výrazně usnadnil práci s~databází systému.

% Nadřazený systém pracuje na základě modelu \textit{client/server}. Klienty jsou v tomto případě podřízené systémy přistupující k API a uživatel používající webové rozhraní. neco vo tim planovaci ale ne mmoc dlouhyho

K~výsledné aplikaci byla vytvořena sada automatizovaných testů, které ověřují funkčnost jednotlivých částí (například přihlášování do webového rozhraní). Tyto testy slouží především k~dalšímu rozšiřování aplikace, neboť s~jejich pomocí lze rychle ověřit, zda přidaný kód neporušil nějakou již implementovanou funkcionalitu. Pro další testování byl také vytvořen jednoduchý simulátor podřízeného systému.

V~rámci testování byla aplikace nasazena na virtuálním serveru služby Amazon EC2\footnote{Aplikace je do konce června 2018 dostupná na \url{https://demo-garaze.tk}.}. Účelem bylo otestovat proces nasazování aplikace. Nasazenou aplikaci (včetně simulátoru) je také možné použít k~demonstraci funkce nadřazeného systému.

Výsledná aplikace nadřazeného systému byla poté úspěšně nasazena i na Raspberry Pi 3, čímž vzniklo zařízení, které bylo cílem této diplomové práce.

\subsection{Zdrojové kódy}

Při vývoji aplikace nadřazeného systému byl použit verzovací systém Git. Repozitář se zdrojovým kódem je volně k~dispozici (pod licencí LGPL) na serveru Github -- \url{https://github.com/ggljzr/mi-dip-impl}. Repozitář také obsahuje testy, předpřipravené konfigurační soubory a stručný návod k~nasazení aplikace na Raspberry Pi.

%\begin{itemize}
%    \item implementace
%    \begin{itemize}
%        \item spousta pouzitejch technologii/frameworku
%        \item z~tech zajimavejch flask, jinja, apscheduler
%    \end{itemize}
%\end{itemize}