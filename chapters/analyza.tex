\chapter{Analýza a návrh}

\section{Výběr komunikačního protokolu}

Nadřazený systém bude se svými klienty (monitorovací zařízení v~jednotlivých garážích) komunikovat přes WiFi nebo Ethernet. Základem komunikace bude TCP/IP protokol, je však potřeba zvolit vhodný protokol z~aplikační vrstvy OSI modelu, který na něm bude stavět.

\subsection{Vlastní protokol}

Jedna z~možností je implementovat vlastní protokol pomocí TCP/IP socketů. Toto řešení se mi však nezdá příliš vhodné, neboť nepřináší žádné významné výhody, naopak se s~ním pojí řada komplikací.

Pro vlastní protokol by bylo nutné vytvořit robustní server, který zvládá obsluhu více klientů najednou. Dále by vzhledem k~citlivosti přenášených dat bylo nutné implementovat nějakou formu šifrování. Tyto velmi obsáhlé problémy přitom řeší většina dnešních protokolů.

Další nevýhodou je nutnost implementace klientské části protokolu při vytváření nových zařízení spravovaných nadřazeným systémem. To do jisté míry omezuje jeho rozšiřitelnost.

\subsection{\texttt{HTTPS}}

Další možnost je využít ke komunikaci protokol HTTPS. V~tomto případě by klienti komunikovali se sytémem pomocí HTTP metod jako například \verb|get| nebo \verb|post|.

Jelikož součástí požadavků na systém je i webové uživatelské rozhraní, bude v~každém případě nutné použít webový server pro jeho provoz. Ten by pak bylo možné využít i k~poskytnutí API pro komunikaci systému s~garážovými čidly.

Vhodný webový server (jako například \textit{Nginx}) zajistí vícevláknovou obsluhu všech klientů. HTTPS se také postará o~kryptografické zabezpeční přenášených dat, je však nutné získat certifikát k~ověření pravosti serveru. 

Lze použít například certifikáty nadace Let's Encrypt, které jsou poskytovány  zdarma. Kromě toho dodává Let's Encrypt také automatizačního klienta \textit{Certbot} \cite{certbot} pro snadné nasazení a aktualizaci jejich certifikátů.

Certifikát pro provoz HTTPS bude potřeba zajistit i v~případě, že komunikace s~klienty nebude postavena na tomto protokolu. Je totiž nutné také zabezpečit webové rozhraní, například kvůli ověření identity uživatele. Nutnost pořízení certifikátu tedy nepředstavuje nevýhodu oproti jiným protokolům.

API realizované pomocí HTTPS je poměrně snadno rozšiřitelné. Pro nově implementovanou operaci stačí definovat URL a případně formát přenášených dat.

Další výhodou je snadná implementace na straně klienta, tedy garážového čidla. Knihovny umožňující vytvářet HTTP požadavky jsou dostupné na většině populárních platforem jako například \textit{Arduino} (s~Ethernet shieldem, oficiální knihovna \textit{EthernetClient} \cite{ard_web}) nebo \textit{ESP8266} (knihovna \textit{esp8266wifi} \cite{esp_web}).

\subsection{Protokol \texttt{MQTT}}

\section{Výběr platformy}

Pro realizaci systému je nutné zvolit vhodnou platformu. Jelikož je cílem práce vytvořit fyzické zařízení, rozhodl jsem se jako základ použít některý z~jednodeskových počítačů, které jsou v~dnešní době na trhu. Tyto počítače bývají cenově velmi dostupné a zároveň poskytují dostatečný výkon a podporu pro provoz systému.

Při výběru počítače byla nejdůležitejším kritériem podpora softwaru potřebného k~implementaci monitorovacího systému.